\documentclass[12pt, a4paper]{article}

% ==================================================
% 1. 基础宏包设置
% ==================================================
\usepackage[UTF8]{ctex}     % 中文支持
\usepackage{geometry}       % 页面边距设置
\geometry{left=2.5cm, right=2.5cm, top=2.5cm, bottom=2.5cm}
\usepackage{amsmath, amssymb, amsthm} % 数学公式与符号
\usepackage{graphicx}       % 插入图片
\usepackage{hyperref}       % 超链接与书签
\usepackage{fancyhdr}       % 页眉页脚定制
\usepackage{xcolor}         % 颜色支持

% ==================================================
% 2. 重点公式高亮框 (使用 tcolorbox)
% ==================================================
\usepackage{tcolorbox}
\newtcolorbox{importantbox}[1][]{
  colback=blue!5!white,    % 背景色:极淡蓝
  colframe=blue!75!black,  % 边框色:深蓝
  fonttitle=\bfseries,
  title={#1}               % 标题(可选)
}

% ==================================================
% 3. 数学环境定义 (定理、定义、例题)
% ==================================================
% 定义定理样式
\newtheoremstyle{mystyle}% name
  {10pt}%      Space above
  {10pt}%      Space below
  {\itshape}%  Body font (斜体,强调内容)
  {}%          Indent amount
  {\bfseries}% Theorem head font (粗体)
  {.}%         Punctuation after theorem head
  { }%         Space after theorem head
  {}%          Theorem head spec

\theoremstyle{mystyle}
\newtheorem{theorem}{定理}[section]  % 按章节编号
\newtheorem{lemma}[theorem]{引理}
\newtheorem{proposition}[theorem]{命题}

\theoremstyle{definition}
\newtheorem{definition}{定义}[section]
\newtheorem{example}{例题}[section]
\newtheorem{exercise}{练习}[section]
\renewcommand{\theexercise}{\arabic{exercise}}

\theoremstyle{remark}
\newtheorem*{remark}{注} % 不编号的备注

% ==================================================
% 3.1 解答环境(用于计算题,不显示证毕符号)
% ==================================================
\newenvironment{solution}[1][解]{%
  \begin{proof}[#1]\renewcommand{\qedsymbol}{}%
}{%
  \end{proof}
}

% ==================================================
% 4. 页眉页脚设置
% ==================================================
\pagestyle{fancy}
\fancyhf{} 
\lhead{数学复习}           % 左页眉
\rhead{\leftmark}             % 右页眉:当前章节名
\cfoot{\thepage}              % 页脚:页码

% ==================================================
% 5. 文档信息
% ==================================================
\title{\textbf{华科工程数学复习笔记}}
\author{penguin chick}
\date{\today}

% ==================================================
% 正文开始
% ==================================================
\begin{document}

\maketitle
\tableofcontents % 生成目录
\newpage

% --------------------------------------------------
% 第一章示例
% --------------------------------------------------
\section{矩阵论}

\subsection{作业学习}

\begin{exercise}[求坐标]%第一题
在$R^4$中,求向量x在基底$\{\alpha_1, \alpha_2, \alpha_3, \alpha_4\}$上的坐标。
$\alpha_1= \begin{bmatrix} 1 & 1 & 1 & 1 \end{bmatrix}^T$,
$\alpha_2= \begin{bmatrix} 1 & 1 & -1 & -1 \end{bmatrix}^T$,
$\alpha_3= \begin{bmatrix} 1 & -1 & 1 & -1 \end{bmatrix}^T$,
$\alpha_4= \begin{bmatrix} 1 & -1 & -1 & 1 \end{bmatrix}^T$,
$x= \begin{bmatrix} 1 & 2 & 1 & 1 \end{bmatrix}^T$。
\end{exercise}

\begin{solution}
求向量x在基底${\alpha_1, \alpha_2, \alpha_3, \alpha_4}$下的坐标,即求一组数
$k_1, k_2, k_3, k_4$,使得:$x = k_1\alpha_1 + k_2\alpha_2 + k_3\alpha_3 + k_4\alpha_4$,
随后转为矩阵方程求解即可:$Ak = x$,A是由基底形成的矩阵一定可逆(正交线性无关),将x左乘
A的逆矩阵即可得到k。
$A = \begin{bmatrix} 1 & 1 & 1 & 1 \\ 1 & 1 & -1 & -1 \\ 1 & -1 & 1 & -1 \\ 
  1 & -1 & -1 & 1 \end{bmatrix}$
$A^{-1} = \begin{bmatrix} 1/4 & 1/4 & 1/4 & 1/4 \\ 1/4 & 1/4 & -1/4 & -1/4 \\ 1/4 & 
  -1/4 & 1/4 & -1/4 \\ 1/4 & -1/4 & -1/4 & 1/4 \end{bmatrix}$
随后$k = A^{-1}x = \begin{bmatrix} 5/4 \\ 1/4 \\ -1/4 \\ -1/4 \end{bmatrix}$
\end{solution}
\begin{remark}
注意记得公式$Ak = x$即可
\end{remark}

\begin{importantbox}[核心公式:根据基底求向量坐标,坐标右乘,基底左乘]
\begin{equation}
  Ak = x
\end{equation}
其中A是以基底形成的矩阵,k是坐标向量,x是目标向量。
\end{importantbox}

\begin{exercise}[求基变换矩阵与有相同坐标向量]%第二题
在$R^3$中,有两个基:$B_1=\left\{ \begin{pmatrix} 1 \\ -1 \\ 0 \end{pmatrix}, 
\begin{pmatrix} 0 \\ 1 \\ -1 \end{pmatrix}, \begin{pmatrix} 0 \\ 0 \\ 1 
\end{pmatrix}\right\}$,$B_2=\left\{ \begin{pmatrix} 1 \\ -1 \\ 1 \end{pmatrix}, 
\begin{pmatrix} 0 \\ 1 \\ 1 \end{pmatrix}, \begin{pmatrix} 1 \\ 0 \\ 1 \end{pmatrix}
\right\}$。

(1)求$B_1$到$B_2$的基变换矩阵;

(2)求在$B_1$,$B_2$下有相同坐标向量的向量。
\end{exercise}

\begin{solution}
(1)求基变换矩阵只需要套公式:$B_2 = B_1P$,P即为所求基变换矩阵。
$P = B_1^{-1}B_2 = \begin{bmatrix} 1 & 0 & 0 \\ 1 & 1 & 0 \\ 1 & 1 & 1 \end{bmatrix}
\begin{bmatrix} 1 & 0 & 1 \\ -1 & 1 & 0 \\ 1 & 1 & 1 \end{bmatrix} = 
\begin{bmatrix} 1 & 0 & 1 \\ 0 & 1 & 1 \\ 1 & 2 & 2 \end{bmatrix}$ 

(2)求相同坐标向量:$x = B_1k = B_2k = B_1Pk$进而可推出$k = Pk$有$(P-I)k = 0$转为一个解
齐次方程解的问题。$P-I = \begin{bmatrix} 0 & 0 & 1 \\ 0 & 0 & 1 \\ 1 & 2 & 1 \end{bmatrix}$
后续就是解线性方程组,得到解空间,即为所求。
\end{solution}

\begin{remark}
注意记得公式$B_2 = B_1P$即可
\end{remark}

\begin{importantbox}[核心公式:基变换矩阵]
\begin{equation}
  B_2 = B_1P
\end{equation}
\end{importantbox}

\begin{exercise}[求span基和维数]%第三题
在$R^4$中的向量:$x_1= \begin{pmatrix} 1 \\ 2 \\ 1 \\ 0 \end{pmatrix},
\quad x_2= \begin{pmatrix} -1 \\ 1 \\ 1 \\ 1 \end{pmatrix},\quad x_3= 
\begin{pmatrix} 2 \\ -1 \\ 0 \\ 1 \end{pmatrix},\quad x_4= \begin{pmatrix} 
-1 \\ -1 \\ 3 \\ 7 \end{pmatrix}$分别张成子空间$w_{1}=span\{x_{1},x_{2}\}$,
$w_{2}=span\{x_{3},x_{4}\}$。求$w_1+w_2$和$w_{1}\cap w_{2}$的基和维数。
\end{exercise}

\begin{solution}
先利用高斯消元法找线性无关的向量个数来构造基底。$w_1+w_2$的维数就是$x_1,x_2,x_3,x_4$
的秩($w_1+w_2 = span\{x_1,x_2,x_3,x_4\}$)。同时$w_1+w_2$的基就是这些向量中线性无关的
向量组成的集合。

接下来找$w_{1}\cap w_{2}$的基和维数。利用维数公式:$dim(w_1+w_2) = dim(w_1)+dim(w_2)-
dim(w_1\cap w_2)$,易见$w_1$和$w_2$的维数分别为2,所以预期的$w_{1}\cap w_{2}$维数为1。
又$\xi\in w_1\cap w_2$,所以$\xi$可以同时用$w_1$和$w_2$的基底来进行表示,只需要找到一个解
即可证明$\xi$的维数为1。设$\xi = k_1x_1 + k_2x_2 = k_3x_3 + k_4x_4$,转化为矩阵方程求解问题。
$k_1x_1 + k_2x_2 - k_3x_3 - k_4x_4 = 0$可推出:$\begin{bmatrix} 1 & -1 & -2 & 1 \\ 
2 & 1 & 1 & 1 \\ 1 & 1 & 0 & -3 \\ 0 & 1 & -1 & -7 \end{bmatrix} \begin{bmatrix} 
k_1 \\ k_2 \\ k_3 \\ k_4 \end{bmatrix} = 0$进而转为解线性方程组,得到解空间,即为所求基底。
\end{solution}

\begin{remark}
注意记得公式$w_1+w_2$和$w_1\cap w_2$的内涵即可
\end{remark}

\begin{exercise}[证明解空间]%第四题
设$V_1$和$V_2$分别是齐次方程组$x_1+x_2+...+x_n = 0$和$x_1 =  x_2 = x_n$的解空间,证明
$R^n=V_1\oplus V_2$。
\end{exercise}

\begin{proof}
子空间直和需要满足两个条件:$V_1\cap V_2=\{\mathbf{0}\}$和$\dim(V_1)+\dim(V_2)=n$。
第一个方程组显然是个n个未知数,1个方程的齐次线性方程组,秩为1,所以根据$\dim(V)=n-
\mathrm{rank}(A)=n-1$可知$\dim(V_1)=n-1$。第二个方程组的任意向量$x$可以表示为:
$x = k\begin{bmatrix} 1 & 1 & ... & 1 \end{bmatrix}^T$,只有一个自由变量,所以$\dim(V_2)=1$。
所以$\dim(V_1)+\dim(V_2)=(n-1)+1=n$。接下来验证交集只有零向量,设$x\in V_1\cap V_2$,
则$x$同时满足两个方程组,所以$x$必须满足$x_1=x_2=...=x_n$且$x_1+x_2+...+x_n=0$,
设$\mathbf{x}=(k,k,\cdots,k)^T$,则$nk = 0$,所以$k = 0$,所以$\mathbf{x}=(0,0,\cdots,0)
^T=\mathbf{0}$,即可说明$V_1\cap V_2=\{\mathbf{0}\}$。
最后,$\dim(V_1+V_2) = \dim(V_1)+\dim(V_2)-\dim(V_1\cap V_2) = n$,又由于$V_1+V_2$是$R^n$的
子空间,其维数为n,所以:$V_1+V_2 = R^n$,综上所述,$R^n=V_1\oplus V_2$得证。
\end{proof}

\begin{remark}
注意解空间维数公式:$\dim(V)=n-\mathrm{rank}(A)=n-1$
\end{remark}

\begin{exercise}[证明解空间]
设$\{\alpha_1, \alpha_2, \alpha_3, \alpha_4\}$是$R^4$的一个基,$V_1 = 
span\{2\alpha_1+\alpha_2,\alpha_1\}$,$V_2 = span\{\alpha_3-\alpha_4,\alpha_1+\alpha_4\}$,
证明:$R^4=V_1\oplus V_2$。
\end{exercise}

\begin{proof}
只需要类似题4,证明$V_1\cap V_2=\{\mathbf{0}\}$和$\dim(V_1)+\dim(V_2)=4$。
求$\dim(V_1)+\dim(V_2)=4$只需要证明$\dim(V_1)=2$和$\dim(V_2)=2$即可。由于
$\alpha_1, \alpha_2$线性无关,所以$2\alpha_1+\alpha_2$和$\alpha_1$线性无关,所以
$\dim(V_1)=2$。同理,$\alpha_3-\alpha_4$和$\alpha_1+\alpha_4$线性无关,所以$\dim(V_2)=2$。
所以$\dim(V_1)+\dim(V_2)=2+2=4$。接下来验证交集只有零向量,设$x\in V_1\cap V_2$,
则$x$同时满足两个子空间的生成条件,所以$x$必须满足:
$x = k_1(2\alpha_1+\alpha_2) + k_2\alpha_1 = k_3(\alpha_3-\alpha_4) + 
k_4(\alpha_1+\alpha_4)$,后续转化为求线性方程组的解即可,解得唯一解为零向量。
综上所述,$R^4=V_1\oplus V_2$。
\end{proof}

\begin{exercise}[证明线性变换]
设$T_1$是$V_n$到$V_m$的线性变换,$T_2$是$V_m$到$V_r$的线性变换,定义$V_n$到$V_r$的变换
$T_2\bullet T_1$为:$(T_2\bullet T_1)\alpha=T_2(T_1\alpha),\forall\alpha\in V^n$,
证明:$T_2\bullet T_1$是线性变换。
\end{exercise}

\begin{proof}
要证明一个变换是线性变换,需要验证加法保持性和数乘保持性$\alpha,\beta\in V^n,T(\alpha+\beta)=
T(\alpha)+T(\beta)$,同时有$T(k\alpha) = kT(\alpha)$。接下来分别证明:

$\begin{aligned} (T_2\bullet T_1)(\alpha+\beta) & =
T_2\left(T_1(\alpha+\beta)\right) \\ & =T_2\left(T_1\alpha+T_1\beta\right) \\ & =
T_2(T_1\alpha)+T_2(T_1\beta) \\ & =(T_2\bullet T_1)\alpha+(T_2\bullet T_1)\beta 
\end{aligned}$

$\begin{aligned} (T_2\bullet T_1)(k\alpha) & =T_2\left(T_1(k\alpha)\right) \\ & =
T_2\left(kT_1\alpha\right) \\ & =kT_2(T_1\alpha) \\ & =k(T_2\bullet T_1)\alpha 
\end{aligned}$

所以是线性变换。
\end{proof}

\begin{remark}
只需要掌握加法性质和乘法性质即可
\end{remark}

\begin{exercise}[求线性变换矩阵]
已知$R^3$的线性变换T在基$B_1=\left\{ \begin{bmatrix} -1 \\ 1 \\ 1 \end{bmatrix}, 
\begin{bmatrix} 1 \\ 0 \\ -1 \end{bmatrix}, \begin{bmatrix} 0 \\ 1 \\ 1 \end{bmatrix}
\right\}$下的矩阵是$\begin{bmatrix} 1 & 0 & 1 \\ 1 & 1 & 0 \\ -1 & 2 & 1 \end{bmatrix}$,
求T在基$B_2=\left\{ \begin{bmatrix} 1 \\ -1 \\ 1 \end{bmatrix}, \begin{bmatrix} 1 \\ 0 
  \\ 1 \end{bmatrix}, \begin{bmatrix} 1 \\ 1 \\ 2 \end{bmatrix}\right\}$下的矩阵。
\end{exercise}

\begin{solution}
先求基变换矩阵$P$,满足$B_2 = B_1P$,所以$P = B_1^{-1}B_2$,然后根据相似标准型公式即可求解。

$(\beta_1,\beta_2,\beta_3)=(\alpha_1,\alpha_2,\alpha_3)P$

$B=P^{-1}AP$

其中$\alpha, \beta$分别为基$B_1, B_2$,A为T在基$B_1$下的矩阵,B为T在基$B_2$下的矩阵。
\end{solution}

\begin{remark}
注意相似求解公式$B=P^{-1}AP$即可
\end{remark}

\begin{exercise}[证明可逆变换]
V的变换T称为可逆的,如果存在V的变换S使$T\bullet S=S\bullet T=I$。这时S称为T的逆变换,记为
$T^{-1}$。证明:

(1)若线性变换T是可逆的,则$T^{-1}$也是线性变换;

(2)T的特征值一定不为零;

(3)若$\lambda$是T的特征值,则$\lambda^{-1}$是$T^{-1}$的特征值。又若T在基B下的矩阵为A,
则$T^{-1}$在B下的矩阵为什么?
\end{exercise}

\begin{proof}
(1)证明线性变换两个性质即可:设$\alpha, \beta$是V中的任意两个向量,则存在唯一的向量$x,y$使得
$T(x) = \alpha, T(y) = \beta$,所以:$x = T^{-1}(\alpha), y = T^{-1}(\beta)$。

接下来验证加法保持性:$T(x+y)=T(x)+T(y)=\alpha+\beta$,
$T^{-1}(T(x+y))=T^{-1}(\alpha+\beta)$,$x+y=T^{-1}(\alpha+\beta)$,
$T^{-1}(\alpha)+T^{-1}(\beta)=T^{-1}(\alpha+\beta)$

接下来验证乘法保持性:$T(kx)=kT(x)=k\alpha$,$T^{-1}(T(kx))=T^{-1}(k\alpha)$,
$kx=T^{-1}(k\alpha)$,$kT^{-1}(\alpha)=T^{-1}(k\alpha)$

所以$T^{-1}$也是线性变换。

(2)利用反证法,假设$\lambda = 0$是T的一个特征值,则存在一个非零特征向量使得:
$T(\alpha) = 0\alpha = 0$,由于T是线性变换,$T(0) = 0$恒成立,又是可逆变换必须为单射,所以
$\alpha = 0$,与$\alpha$为非零向量矛盾,所以$\lambda \neq 0$。

(3)$T(\xi)=\lambda\xi$,$T^{-1}(T(\xi))=T^{-1}(\lambda\xi)$,$\xi=T^{-1}(\lambda\xi)$,
利用(1)中的数乘保持性,$\xi=\lambda T^{-1}(\xi)$,即$T^{-1}(\xi)=\frac{1}{\lambda}\xi$证毕。

由于T在基B下的矩阵为A,所以:$T(B)=BA$,$T^{-1}\left(T(B)\right)=T^{-1}(BA)=B$,所以
$T^{-1}(B)=BA^{-1}$,于是$T^{-1}$在B下的矩阵为$A^{-1}$。
\end{proof}

\begin{remark}
记得线性变换的特征值和特征向量的关系公式:$T(\alpha) = \lambda \alpha$

同时记得T在基下的矩阵公式:$T(B)=BA$
\end{remark}

\begin{exercise}[求特征值和特征向量]
设T是复数域上线性空间V的线性变换,已知V的基B和T在B下的矩阵A如下,求T的特征值和特征向量:

(1)$B= \begin{bmatrix} \alpha_1 & & \alpha_2 & & \alpha_3 \end{bmatrix},\quad A= 
\begin{pmatrix} 2 & & 2 & & -1 \\ -1 & & -1 & & 1 \\ -1 & & 2 & & 2 \end{pmatrix}$

(2)$B=\left\{ \begin{pmatrix} 1 \\ -1 \end{pmatrix}, \begin{pmatrix} 0 \\ 1 
\end{pmatrix}\right\},\quad A= \begin{pmatrix} 3 & 4 \\ 5 & 2 \end{pmatrix}$
\end{exercise}

\begin{solution}
先求A的特征值$|\lambda I-A|= \begin{vmatrix} \lambda-2 & -2 & 1 \\ 1 & \lambda+1 & -1 
\\ 1 & -2 & \lambda-2 \end{vmatrix}$,解得$\lambda_1=1,\quad\lambda_2=-1,\quad\lambda_3=3$
,这即是T的特征值,然后再求A的特征向量。求出坐标向量之后,
再利用基底矩阵B左乘坐标向量即可得到特征向量。

$\xi_1=1\alpha_1+0\alpha_2+1\alpha_3=\alpha_1+\alpha_3$,

$\xi_2=\alpha_1-\alpha_2+\alpha_3$,

$\xi_3=-\alpha_1+\alpha_2+3\alpha_3$。

对于第二问就是类似求解手段。
\end{solution}

\begin{remark}
三步走,先求A的特征值(T的特征值),然后求T的特征向量的坐标向量(A的特征向量),然后还原特征向量。
\end{remark}

\begin{exercise}[求最小多项式]
求下列方阵的最小多项式:

$\begin{pmatrix} 3 & 1 \\ & 3 & 1 \\ & & 3 \end{pmatrix},\quad \begin{pmatrix} 
3 & 1 \\ -4 & -1 \\ & & 2 & 1 \\ & & -1 & 0 \end{pmatrix},\quad \begin{pmatrix} 
3 & 1 & 1 & 1 \\ -4 & -1 & -1 & 1 \\ & & 2 & 1 \\ & & -1 & 0 \end{pmatrix},\quad 
\begin{pmatrix} 2 & 2 \\ 1 & 3 \\ & & 4 & 1 \\ & & & 4 \end{pmatrix}$
\end{exercise}

\begin{solution}
记住结论,对于k阶约当块,最小多项式等于其特征多项式;分块对角矩阵的最小多项式是各对角块最小多项式
的最小公倍数;普通矩阵就直接整体求就行。
\end{solution}

\begin{remark}
最小多项式求解方式,找到$A-\lambda I$的最小k次方为0的k即可。可对角化的话,最小多项式没有重根。
\end{remark}

\begin{exercise}[判断是否可对角化]
满足下述条件的方阵A是否可对角化?

(1)A是幂零矩阵;

(2)$A^k=I(k\geq2)$

(3)$A^{2}+A=2I$
\end{exercise}

\begin{solution}
(1)幂零矩阵满足性质:$A^m = 0$且只有唯一特征值0,所以非零幂零矩阵不可对角化。

(2)最小多项式无重根即可对角化,$A^k - I = 0$可推出$A^\lambda - 1 = 0$,原函数和导数没有公共根
就能推出原函数无重根。

(3)求出特征根发现无重根,所以可对角化。$A^2+A-2I=0$推出$\lambda^2+\lambda-2=0$。
\end{solution}

\begin{remark}
由矩阵多项式可以往特征值多项式上推导。
\end{remark}

\begin{exercise}[求矩阵的幂级数]
已知$A= \begin{pmatrix} 2 & -1 & -2 \\ -1 & 2 & 2 \\ 0 & 0 & 1 \end{pmatrix}$,求:

$g(A)=A^{8}-9A^{6}+A^{4}-3A^{3}+4A^{2}+I$。
\end{exercise}

\begin{solution}
先求A的特征多项式:$(\lambda-1)^2(\lambda-3)$,接下来确定A的最小多项式:
只需要判断$A-I$即可,判断$A-I$线性无关的特征向量个数,发现等于2,于是可以说明对应于$\lambda = 1$的
指数是1,所以最小多项式为:$(\lambda-1)(\lambda-3)$。
这意味着矩阵A满足:$A^2-4A+3I=0$,则$g(\lambda)=q(\lambda)m(\lambda)+r(\lambda)$,其中
$r(\lambda)$是余式,次数小于2,设为$r(\lambda)=a\lambda+b$。带入矩阵A,因为$m(A)=O$,所以
$g(A)=q(A)\cdot O+r(A)=aA+bI$,接下来需要确定系数a和b。利用特征值$\lambda=1$和$\lambda=3$代入
方程$g(\lambda)$中即可。解得化简后的结果为:$g(A)=21A-26I$,然后再代入A解最后结果。
\end{solution}

\begin{remark}
最小多项式求解,判断每个特征值的几何重数和代数重数来确定最小多项式的指数大小问题。
\end{remark}

\begin{exercise}[求约旦标准型]
$f(\lambda)$和$m(\lambda)$分别表示矩阵A的特征多项式和最小多项式,确定A的可能的Jordan标准型:

$(1).\quad f(\lambda)=(\lambda-2)^{4}(\lambda-3)^{2},\quad m(\lambda)=(\lambda-2)^{2}
(\lambda-3)$

$(2).\quad f(\lambda)=(\lambda-3)^{3}(\lambda+2)^{3},\quad m(\lambda)=
(\lambda-3)^{2}(\lambda+2)$
\end{exercise}

\begin{solution}
(1)特征值$\lambda = 2$时为四重根,但是由最小多项式可知其最大约当块为2,所以可能组合为2+2或2+1+1
特征值$\lambda = 3$时为二重根,且最小多项式可知其最大约当块为2,所以可能组合为2

(2)同理
\end{solution}

\begin{remark}
分辨特征多项式和最小多项式中关于约当块的知识点。
\end{remark}

\begin{exercise}[求可逆矩阵P使得成为Jordon标准型]
求可逆矩阵P,使得$P^{-1}AP$为Jordan矩阵,其中$A= \begin{pmatrix} 3 & 1 & -1 \\ 2 & 2 & -1 
\\ 2 & 2 & 0 \end{pmatrix}$
\end{exercise}

\begin{solution}
先求特征值,特征多项式为:$(\lambda-1)(\lambda-2)^2$
其中$\lambda=1$只有一个1阶约当块;$\lambda=2$几何重数为1,只有一个线性无关的特征向量,对应
一个Jordan块,所以块是二阶的。于是jordan标准型为:$J= \begin{pmatrix} 1 & 0 & 0 \\ 0 & 2 & 1 
\\ 0 & 0 & 2 \end{pmatrix}$

解出$\lambda=1$时的特征向量为:$(A-I)\alpha_1=0$,$\alpha_1=(1,0,2)^T$

解出$\lambda=2$时的特征向量为:$(A-2I)\alpha_2=0$,$(A-2I)\alpha_3=\alpha_2$,$\alpha_2=
(1,1,2)^T$,$\alpha_3=(0,0,-1)^T$

于是构造得到变换矩阵:$\boldsymbol{P}= \begin{pmatrix} 1 & 1 & 0 \\ 0 & 1 & 0 \\ 2 & 2 & 
-1 \end{pmatrix}$,进而得到:$P^{-1}AP=J=\begin{pmatrix} 1 & 0 & 0 \\ 0 & 2 & 1 \\ 0 & 0 
& 2 \end{pmatrix}$
\end{solution}

\begin{remark}
先判断约旦标准型形式,再一步步求变换矩阵P。
\end{remark}

\begin{exercise}[求Jordon标准型]
设$V^4$是由函数$e^{x},xe^{x},x^{2}e^{x},e^{2x}$张成的线性空间,求$V^4$的线性变换
$D=\frac{d}{dx}$的Jordan标准型。
\end{exercise}

\begin{solution}
先取基:$\alpha_1=e^x,\alpha_2=xe^x,\alpha_3=x^2e^x,\alpha_4=e^{2x}$,然后计算对各个基的线性
变换:$\begin{aligned} & 1.D(\alpha_1)=(e^x)^{\prime}=e^x=\mathbf{1}\alpha_1 \\ & 
2.D(\alpha_2)=(xe^x)^{\prime}=e^x+xe^x=\alpha_1+\alpha_2 \\ & 3.D(\alpha_3)=(x^2e^x)^
{\prime}=2xe^x+x^2e^x=\boldsymbol{2}\alpha_2+\alpha_3 \\ & 4.D(\alpha_4)=(e^{2x})^
{\prime}=2e^{2x}=\mathbf{2}\alpha_4 \end{aligned}$
可以将以上结果写为矩阵:$A= \begin{pmatrix} 1 & 1 & 0 & 0 \\ 0 & 1 & 2 & 0 \\ 0 & 0 & 1 & 
0 \\ 0 & 0 & 0 & 2 \end{pmatrix}$

A为上三角矩阵,$\lambda_1=1$和$\lambda_2=2$分别出现3次和1次,特征值为1时几何重数为1,说明只有
一个约当块,所以约当标准型为:$J=\begin{pmatrix} 1 & 1 & 0 & 0 \\ 0 & 1 & 1 & 0 \\ 0 & 0 
& 1 & 0 \\ 0 & 0 & 0 & 2 \end{pmatrix}$
\end{solution}

\begin{remark}
核心就是基变换公式:$T(B)=BA$
\end{remark}

\begin{exercise}[求微分方程组的解]
设$\mathrm{A}= \begin{pmatrix} -2 & 1 & 0 \\ -4 & 2 & 0 \\ 1 & 0 & 1 \end{pmatrix}$,
$f(t)= \begin{pmatrix} 1 \\ 2 \\ e^t-1 \end{pmatrix}$,求微分方程组$x'(t)=Ax(t)+f(t)$
满足初始条件$x(0)= \begin{pmatrix} 1 \\ 1 \\ -1 \end{pmatrix}$的解。
\end{exercise}

\begin{solution}
先求矩阵的最小多项式,具体操作流程和上面的题一样,求出来发现为:$m_A(\lambda)=\lambda^2
(\lambda-1)$。
接下来利用插值多项式求矩阵指数:$e^{At}$,设$g(\lambda)=c_0+c_1\lambda+c_2\lambda^2$,
得到对应方程组:$\begin{cases} g(0)=c_0=e^{0\cdot t}=1 \\ g^{\prime}(0)=c_1=(te^{t\lambda})
|_{\lambda=0}=t \\ g(1)=c_0+c_1+c_2=e^{1\cdot t}=e^t & \end{cases}$

解得:$e^{At}=c_0I+c_1A+c_2A^2=I+tA+(e^t-1-t)A^2$

然后利用非齐次方程组的解公式:$x(t)=e^{At}x(0)+\int_0^te^{A(t-\tau)}f(\tau)d\tau$

分别带入已有数值,解得初始解为:$x(t)= \begin{pmatrix} 1 \\ 1 \\ (t-1)e^t \end{pmatrix}$
\end{solution}

\begin{remark}
这个题不算考纲内,注意非齐次解公式$x(t)=e^{At}x(0)+\int_0^te^{A(t-\tau)}f(\tau)d\tau$
\end{remark}

\begin{exercise}[求满秩分解]
求下列矩阵的满秩分解:$(1) \begin{pmatrix} 1 & 2 & 3 & 0 \\ 0 & 2 & 1 & -1 \\ 1 & 0 & 2 & 1 
\end{pmatrix};(2) \begin{pmatrix} 0 & 0 & 1 \\ 2 & 1 & 1 \\ 2i & i & 1 \end{pmatrix}.$
\end{exercise}

\begin{solution}
满秩分解是指将A分解为两个矩阵的乘积:$A=FG$,其中$F$列满秩,$G$行满秩。
具体操作流程为:对矩阵$A$进行初等行变换,化为行最简形矩$B$。$B$中非零行的数量即为秩$r$,
构造$G$:取$B$的前$r$个非零行,构造$F$:取原矩阵$A$中对应于$B$的主元列的那些列。

(1)行最简:$\begin{pmatrix} 1 & 0 & 2 & 1 \\ 0 & 1 & \frac{1}{2} & -\frac{1}{2} \\ 
0 & 0 & 0 & 0 \end{pmatrix}$,取前两行组成$G= \begin{pmatrix} 1 & 0 & 2 & 1 \\ 0 & 1 & 
\frac{1}{2} & -\frac{1}{2} \end{pmatrix}$,观察可知取原矩阵的前两列组成$F= \begin{pmatrix} 
1 & 2 \\ 0 & 2 \\ 1 & 0 \end{pmatrix}$,于是$A=FG= \begin{pmatrix} 1 & 2 \\ 0 & 2 \\ 1 
& 0 \end{pmatrix} \begin{pmatrix} 1 & 0 & 2 & 1 \\ 0 & 1 & \frac{1}{2} & -\frac{1}{2} 
\end{pmatrix}$

(2)类似做法即可
\end{solution}

\begin{remark}
掌握$FG$的构造手法,取最简行矩阵的非零行组成G,取原矩阵的主元列组成F。
\end{remark}

\begin{exercise}[求QR分解]
求下列矩阵的QR分解:$\mathrm{A}= \begin{pmatrix} 1 & 0 & 0 \\ 1 & 1 & 0 \\ 1 & 1 & 1 
\\ 1 & 1 & 1 \end{pmatrix}.$
\end{exercise}

\begin{solution}
QR分解是指将一个列满秩矩阵A分解为$A = QR$,其中:$Q$是列正交矩阵(即$Q^TQ=I$),其列向量为标准
正交基,$R$是可逆的上三角矩阵。

求解步骤:

正交化:将$A$的列向量 $\alpha_1, \alpha_2, \alpha_3$转化为正交向量组$\beta_1, 
\beta_2, \beta_3$。

单位化:将$\beta_i$单位化得到$\eta_i$,这些$\eta_i$构成矩阵$Q$。

构造R:利用公式$R = Q^TA$。

$\alpha_1= \begin{pmatrix} 1 \\ 1 \\ 1 \\ 1 \end{pmatrix},\quad\alpha_2= \begin{pmatrix} 
0 \\ 1 \\ 1 \\ 1 \end{pmatrix},\quad\alpha_3= \begin{pmatrix} 0 \\ 0 \\ 1 \\ 1 
\end{pmatrix}$,$\beta_1=\alpha_1= \begin{pmatrix} 1 \\ 1 \\ 1 \\ 1 \end{pmatrix}$,
$\beta_2=\alpha_2-\frac{\langle\alpha_2,\beta_1\rangle}{\langle\beta_1,\beta_1\rangle}
\beta_1$,$\beta_3=\alpha_3-\frac{\langle\alpha_3,\beta_1\rangle}{\langle\beta_1,
\beta_1\rangle}\beta_1-\frac{\langle\alpha_3,\beta_2\rangle}{\langle\beta_2,\beta_2
\rangle}\beta_2$
然后分别将它们单位化得到Q:$Q= \begin{pmatrix} \frac{1}{2} & -\frac{\sqrt3}{2} & 0 \\ 
\frac{1}{2} & \frac{\sqrt3}{6} & -\frac{\sqrt6}{3} \\ \frac{1}{2} & \frac{\sqrt3}{6} & 
\frac{\sqrt6}{6} \\ \frac{1}{2} & \frac{\sqrt3}{6} & \frac{\sqrt6}{6} \end{pmatrix}$
R再根据$R = Q^TA$得到。
\end{solution}

\begin{remark}
施密特正交化要掌握。
\end{remark}

\begin{exercise}[求奇异值分解]
求下列矩阵的奇异值分解:$(1) \begin{pmatrix} 1 & 1 \\ 1 & 1 \\ 1 & -1 \end{pmatrix}(2) 
\begin{pmatrix} 1 & 0 \\ 0 & 1 \\ 1 & 1 \end{pmatrix}$
\end{exercise}

\begin{solution}
奇异值分解(SVD)定义:对于任意$m\times n$矩阵$A$,可以分解为$A = U\Sigma V^T$,其中
$U$是$m\times m$正交矩阵(列向量为$AA^T$的特征向量);$V$是$n\times n$正交矩阵(列向量为$A^TA$
的特征向量)$\Sigma$是$m\times n$对角矩阵,主对角线元素为奇异值$\sigma_i = \sqrt{\lambda_i}$
($\lambda_i$ 是$A^TA$的特征值)

$A^TA= \begin{pmatrix} 1 & 1 & 1 \\ 1 & 1 & -1 \end{pmatrix} \begin{pmatrix} 1 & 1 \\ 1 
& 1 \\ 1 & -1 \end{pmatrix}= \begin{pmatrix} 3 & 1 \\ 1 & 3 \end{pmatrix}$

接下来求$A^TA$的特征值与奇异值:$\lambda_1 = 4, \lambda_2 = 2$,则$\sigma_1=2,
\sigma_2=\sqrt{2}$,所以$\Sigma= \begin{pmatrix} 2 & 0 \\ 0 & \sqrt{2} \\ 0 & 0 
\end{pmatrix}$,分别解对应两个特征值的特征向量(顺序要注意),解得:$V= \begin{pmatrix} 
\frac{1}{\sqrt{2}} & \frac{1}{\sqrt{2}} \\ \frac{1}{\sqrt{2}} & -\frac{1}{\sqrt{2}} 
\end{pmatrix},\quad V^T= \begin{pmatrix} \frac{1}{\sqrt{2}} & \frac{1}{\sqrt{2}} \\ 
\frac{1}{\sqrt{2}} & -\frac{1}{\sqrt{2}} \end{pmatrix}$

接下来求$U$矩阵,$AA^T$的特征向量即可。最后得到结果。(2)类似求解手段。
\end{solution}

\begin{remark}
掌握这个公式$A = U\Sigma V^T$,U关于$AA^T$,V关于$A^TA$,$\Sigma$关于奇异值($A^TA$特征值下的)。
\end{remark}

\subsection{重难点学习}
重点复习方向:

\begin{enumerate}
\item 线性空间与线性变换。基础知识
\item 线性组合。基础知识
\item 线性相关与线性无关。一组不全为零的数进行系数表示
\item 线性空间的基底,维数,坐标,基底之间的变换。

最大线性无关向量个数称为维数(秩,基底),基不是唯一的。坐标向量公式:$Ak=x$。
两组基之间的变换关系:$B_\beta=B_\alpha C$,C称为过渡矩阵(基变换矩阵)。
\item 子空间,子空间的运算,不变子空间。

$\alpha_1,\alpha_2,\cdots,\alpha_r,1\leq r\leq n$是V的r个线性无关向量,则集合$span\{\alpha_1,
\alpha_2,\cdots,\alpha_r\}:=\{\sum_{i=1}^rk_i\alpha_i,k_i\in F\}$是V的一个子空间,
称为$\alpha_1,\alpha_2,\cdots,\alpha_r$张成的空间。

子空间的交和:$W_1\cap W_2:=\{x:x\in W_1,x\in W_2\}W_1+W_2:=\{\xi_1+\xi_2: \xi_i\in W_i,
i=1,2\}$。且它们都是V的子空间

重要公式:$dim(W_1+W_2)+dim(W_1\cap W_2)=dim(W_1)+dim(W_2)$

子空间的直和:若$W_1+W_2$中任一向量只能唯一的分解为$W_1$中的一个向量与$W_2$中的一个向量之和, 
则$W_1+W_2$称为$W_1$与$W_2$的直和, 记为$W_1\oplus W_2$。直和成立等价条件:$W_1\cap W_2=\{0\}$
和$dim(W_1+W_2)=dim(W_1)+dim(W_2)$
\item 线性变换,线性变换的矩阵表示,线性变换的核与值域。

线性变换具有的性质:$T(\alpha+\beta)=T(\alpha)+T(\beta);T(k\alpha)=kT(\alpha)$,$T\mathcal
{B}_\alpha=\mathcal{B}_\beta A$,A为T在基偶下的矩阵。设T是从$V_n$到$V_m$的线性变换,则:
$\begin{aligned} N(T) & :=\{\forall\alpha\in V_{n}:T\alpha=0\} \\ R(T) & :=\{\beta\in V_
{m}:\beta=T\alpha,\alpha\in V_{n}\} \end{aligned}$分别称为T的核和T的值域。核也被称为零空间,
记作nullT,值域称为值空间,记作rankT。有:$nullT+rankT=n$。它们也是不变子空间,同时T的不变子空间
的交和也是不变子空间。
\item 线性变换的对角阵表示,线性变换的特征子空间,线性变换的对角化

$T\mathcal{B}_\alpha=\mathcal{B}_\alpha A$,$T(\xi)=\lambda\xi$,$\lambda$为T的特征值,$\xi$
为T对应于$\lambda$的特征向量。$T\xi=\lambda_{0}\xi\Longleftrightarrow Ax=\lambda_{0}x$

线性变换T在不同的基下对应的矩阵是不同的,但是矩阵彼此相似:相似的矩阵具有相同的特征值(T的特征值
是由T决定的,和基与变换矩阵的选择无关),相似矩阵具有不同的特征向量,具有相同特征多项式
且和A的特征多项式一致。

求T的特征对的步骤:

$T\mathcal{B}=\mathcal{BA}$求出矩阵A

$f(\lambda)=\det(\lambda I-A)$求出特征值

$(\lambda_iI-A)Y_i$求出A对应特征值的特征向量

$X_i=\mathcal{B}Y_i$求出T对应的特征向量

几何重数小于或等于代数重数
\item Jordan阵

Jordan阵的求法:

求A的特征多项式,求特征值以及特征向量,套公式:
$\begin{cases} (A-\lambda_{i}I)\alpha_{i}=0 \\ (A-\lambda_{i}I)\beta_{1}=\alpha_{i} \\ 
  (A-\lambda_{i}I)\beta_{k}=\beta_{k-1},2\leq k\leq n_{i} \end{cases}$
\item 最小多项式

求法:Jordan阵直接就是最小多项式,多个Jordan阵时,最小多项式为最大的Jordan阵的特征多项式。
可以判断每个特征值的几何重数和代数重数来确定最小多项式的指数大小。对角块矩阵时,最小多项式是每个
对角块的最小多项式的最小公倍数。

m阶方阵AB和n阶方阵BA的非零特征值相同。

A的特征多项式为其零化多项式;$m_T(\lambda)=m_A(\lambda)$
\item 矩阵的分解:满秩分解,SVD分解,正规矩阵(其他老师还有UR分解,三角分解)

LU分解,下三角上三角,先用行化简得到U,再简单计算得到L。一般Ax=b,先LU分解,然后Ly=b解y再Ux=解x。

LDV分解,下三角上三角对角线为1,D为对角矩阵。(LDV就在LU基础上提取对角线)

满秩分解,秩为r的矩阵肯定存在两个秩为r的矩阵,$A=BC$,其中B列满秩,C行满秩。先做行化简,化简到只有
主元上有元素,然后挑出所有主元行构成B,再按照对应列选出原A矩阵中的列构成C。

QR分解,A为列满秩矩阵,A=QR,Q为列向量标准正交,R为主对角线上元素皆正的上三角矩阵。

正规矩阵,$A^HA=A^AH$,即A与A的共轭转置可交换。A是正规矩阵的充要条件是A酉相似于对角阵。还有n个
线性无关的特征向量构成标准正交基。

奇异值(SVD)分解,$A^HA$特征值开根号。A为正规阵时,A的奇异值等于A的特征值的模,A为Hermite矩阵时
A的奇异值等于A的特征值。

$A=U\Sigma V^H$,其中UV均为酉矩阵,$\Sigma$为奇异值对角阵。U求解就是将$A^HA$的特征向量
单位化,V求解就是将$AA^H$的特征向量单位化。$\Sigma$的对角线元素为奇异值,按从大到小排列。
\end{enumerate}

\section{数值分析}

\subsection{作业学习}

\begin{exercise}[拉格朗日插值多项式]%第一题
$f(x)=5x^2(3x-2)(2x+1)$,设$x_i(i=0,1,2,3,4)$为互异节点,$l_i(x)$为对应的4次Lagrange插值基函数
分别求$\sum_{i=0}^{4}l_{i}(x),\sum_{i=0}^{4}f(x_{i})l_{i}(x)$,$f[x_0,x_1,x_2,x_3,x_4]$
\end{exercise}

\begin{solution}
(1)插值基函数性质,对于任何一组互异节点,其对应的Lagrange插值基函数满足单位和性质,即对任意x,
有$\sum l_i(x)=1$

对于y=1的n次Lagrange插值多项式:$L_n(x)=\sum_{i=0}^4y_il_i(x)=\sum_{i=0}^4l_i(x)$
同时余项公式为:$R_n(x)=\frac{f^{(n+1)}(\xi)}{(n+1)!}\prod_{i=0}^n(x-x_i)$,原函数为常数,
余项必然为0,所以$L_n(x)=1$,进而可以得出:$\sum_{i=0}^4l_i(x)=1$

(2)原函数$f(x)=5x^2(3x-2)(2x+1)=5x^2(6x^2-x-2)=30x^4-5x^3-10x^2$为4次多项式,五个节点,算
4次插值多项式,则$L_4(x)=\sum_{i=0}^{4}f(x_{i})l_{i}(x)$必然恒等于$f(x)$,所以
$\sum_{i=0}^{4}f(x_{i})l_{i}(x)=f(x)$

(3)根据差商性质,n次多项式的n阶差商等于其最高次项系数,所以$f[x_0,x_1,x_2,x_3,x_4]=30$
\end{solution}

\begin{remark}
掌握插值多项式$L_n(x)$和$l_n(x)$之间的含义,还有余项公式代表误差$R_n(x)=f(x)-L_n(x)$
\end{remark}

\begin{exercise}[Lagrange和Newton插值多项式]%第二题
$f(x)$给定节点值,求其3次Lagrange和Newton多项式:
$\begin{array} {cccccccccc} x & 1 & 3/2 & 0 & 2 \\ f(x) & 3 & 13/4 & 3 & 5/3 \end{array}$
\end{exercise}

\begin{solution}
(1)对于$x=1$时,$l_0(x)=\frac{(x-3/2)(x-0)(x-2)}{(1-3/2)(1-0)(1-2)}=\frac{x(x-1.5)(x-2)}
{(-0.5)(1)(-1)}=2\mathbf{x(x-1.5)(x-2)}$,对于$x=3/2$时,$l_1(x)=\frac{(x-1)(x-0)(x-2)}
{(3/2-1)(3/2-0)(3/2-2)}=\frac{x(x-1)(x-2)}{(0.5)(1.5)(-0.5)}=-\frac{8}{3}\mathbf{x}
(\mathbf{x}-1)(\mathbf{x}-2)$,对于$x=0$时,$\begin{aligned} & l_2(x)=\frac{(x-1)(x-3/2)
(x-2)}{(0-1)(0-3/2)(0-2)}=\frac{(x-1)(x-1.5)(x-2)}{(-1)(-1.5)(-2)}=-\frac{1}{3}
(\mathbf{x-1})(\mathbf{x-1.5})(\mathbf{x-2}) \end{aligned}$,对于$x=2$时,$l_3(x)=\frac
{(x-1)(x-3/2)(x-0)}{(2-1)(2-3/2)(2-0)}=\frac{x(x-1)(x-1.5)}{(1)(0.5)(2)}=\mathbf{x}
(\mathbf{x}-1)(\mathbf{x}-1.5)$,再分别与$f(x)$值组合即可得到$L_3(x)$。

(2)构造差商表如下:

\begin{align*}
\begin{array}{c|c|c|c|c}
x_i & f(x_i) & \text{一阶差商} & \text{二阶差商} & \text{三阶差商} \\
\hline
0 & 3 & & & \\
& & 0 & & \\
1 & 3 & & -\frac{2}{3} & \\
& & -\frac{4}{3} & & -2 \\
2 & \frac{5}{3} & & -\frac{11}{3} & \\
& & -\frac{19}{6} & & \\
\frac{3}{2} & \frac{13}{4} & & & \\
\end{array}
\end{align*}

解得Newton多项式为:$N_3(x)=3+0(x-0)-2/3(x-0)(x-1)-2(x-0)(x-1)(x-2)$
\end{solution}

\begin{remark}
记住Lagrange插值多项式中$l_n(x)$的求法,Newton插值多项式的差商表构造方法。
\end{remark}

\begin{exercise}[估计插值误差]%第三题
若$p(x)$是$f(x)=e^{x-2}$在节点0,0.1,0.2,...,0.9,1处的10次插值多项式,试估计该插值多项式
在[0,1]上的插值误差。
\end{exercise}

\begin{solution}
$f(x)$导数恒定为$f^{(k)}(x)=e^{x-2}$,所以11阶导为单调增函数,其最大值在x=1处取得:
$M_{11}=\max_{x\in[0,1]}|f^{(11)}(x)|=e^{1-2}=e^{-1}=\frac{1}{e}$

$R_n(x)=f(x)-p(x)=\frac{f^{(n+1)}(\xi)}{(n+1)!}\prod_{i=0}^n(x-x_i)$,代入几个可以计算出的
具体值,然后取绝对值放缩即可得到最后答案。
\end{solution}

\begin{remark}
主要掌握余项公式也就是误差公式:$R_n(x)=f(x)-p(x)=\frac{f^{(n+1)}(\xi)}{(n+1)!}\prod_{i=0}^
n(x-x_i)$
\end{remark}

\begin{exercise}[估计插值误差]
$h(x)$是$f(x)=e^{x-2}$在节点0,0.1,0.2,...,0.9,1处的分段线性插值多项式,试估计该插值多项式
在[0,1]上的插值误差。
\end{exercise}

\begin{solution}
导数与上一题类似。最大值可知。只需要套用分段线性插值公式:

$|R_1(x)|=|f(x)-h(x)|=\left|\frac{f^{\prime\prime}(\xi)}{2}(x-x_i)(x-x_{i+1})\right|,
\quad\xi\in(x_i,x_{i+1})$

有二级结论,分段线性插值的最大误差限公式为:$\max_{x\in[0,1]}|f(x)-h(x)|\leq\frac{h^2}{8}M_2$,
其中$M_2=\max_{x\in[0,1]}|f^{\prime\prime}(x)|$。套值计算即可得最后结果
\end{solution}

\begin{remark}
注意分段线性插值误差计算公式
\end{remark}

\begin{exercise}[Hermite插值多项式]
给定$f(0)=0,f(1)=1,f(0.5)=1,f'(0.5)=2$求$f(x)$的3次Hermite插值多项式,若又知道$f'(1)=-3$,求
$f(x)$的4次Hermite插值多项式。 
\end{exercise}

\begin{solution}
(1)构造3次Hermite差商表,类似于Newton差商表:
(2)构造4次Hermite差商表,同理。
\end{solution}

\begin{remark}
注意Hermite和Newton插商表的不同之处即可
\end{remark}

\begin{exercise}[二次最佳平方逼近]
求$f\left(x\right)=x^{2}+x\cos x$在$[0,\pi]$上的二次最佳平方逼近多项式。
\end{exercise}

\begin{solution}
建立法方程组即可求解。需要求解:$m_{ij}=\int_0^\pi x^{i+j}dx$

选择最简单的基函数族:$\{1,x,x^2\}$,则最佳平方逼近多项式为:$P_2(x)=a_0+a_1x+a_2x^2$
$\begin{aligned} & \mathrm{moo}=\int_0^\pi1dx=\pi \\ & \mathrm{m01}=m_{10}=\int_0^\pi 
xdx=\frac{\pi^2}{2} \\ & \mathrm{m02}=m_{20}=m_{11}=\int_0^\pi x^2dx=\frac{\pi^3}{3} \\ & 
\mathrm{m12}=m_{21}=\int_0^\pi x^3dx=\frac{\pi^4}{4} \\ & \mathrm{m22}=\int_0^\pi x^4dx=
\frac{\pi^5}{5} \end{aligned}$

接下来计算常数项:$b_i=\int_0^\pi(x^2+x\cos x)x^idx$

列出并求解方程组:$\begin{pmatrix} \pi & \pi^2/2 & \pi^3/3 \\ \pi^2/2 & \pi^3/3 & \pi^4/4 
\\ \pi^3/3 & \pi^4/4 & \pi^5/5 \end{pmatrix} \begin{pmatrix} a_0 \\ a_1 \\ a_2 
\end{pmatrix}= \begin{pmatrix} \pi^3/3-2 \\ \pi^4/4-2\pi \\ \pi^5/5-3\pi^2+12 
\end{pmatrix}$即可得到最终答案。
\end{solution}

\begin{remark}
记得法方程组左侧系数和右侧常数项的求解方法。
\end{remark}

\begin{exercise}[证明题,没看懂]
在所有首一的n次多项式中,首一的n次Legendre多项式在[-1,1]上与零的平方误差最小。
\end{exercise}

\begin{solution}
暂无
\end{solution}

\begin{remark}
暂无
\end{remark}

\begin{exercise}[二次多项式拟合]
$\begin{array} {ccccccccccc}x_i & & -2 & & -1 & & 0 & & 1 & & 2 \\ y_i & & 0 & & 1 & & 2 
& & 1 & & 0 \end{array}$

试用二次多项式拟合上述数据,并求拟合误差。$p(x)=ax^2+bx+c$
\end{exercise}

\begin{solution}
待定系数需要满足法方程组:$\begin{aligned} \begin{cases} a\sum x_i^4+b\sum x_i^3+c\sum x_i^2
=\sum x_i^2y_i \\ a\sum x_i^3+b\sum x_i^2+c\sum x_i=\sum x_iy_i \\ a\sum x_i^2+b\sum 
x_i+nc=\sum y_i \end{cases} \end{aligned}$,只需要依次带值求解即可。
\end{solution}

\begin{remark}
记住法方程组公式
\end{remark}

\begin{exercise}[拟合]
$\begin{array} {ccccccc}\mathrm{xi} & 1 & 2 & 3 & 4 \\ \mathrm{yi} & 2 & 1 & 0 & 1 
\end{array}$,求$y=ax+b\mathrm{sin}^{2}\frac{\pi x}{6}$的拟合曲线。
\end{exercise}

\begin{solution}
构建法方程组:$\phi_1(x)=x\mathrm{,}\phi_2(x)=\sin^2\frac{\pi x}{6}$

$\begin{cases} a\sum\phi_1^2(x_i)+b\sum\phi_1(x_i)\phi_2(x_i)=\sum y_i\phi_1(x_i) \\ 
a\sum\phi_1(x_i)\phi_2(x_i)+b\sum\phi_2^2(x_i)=\sum y_i\phi_2(x_i) & \end{cases}$

分别代入具体值解一个方程组即可。
\end{solution}

\begin{remark}
记住法方程组公式
\end{remark}

\begin{exercise}[非线性拟合]
$\begin{array} {cccccccc}x & 1.00 & 1.25 & 1.50 & 1.75 & 2.00 \\ y & 5.10 & 5.79 & 6.53 
& 7.45 & 8.46 \end{array}$,拟合$y=ae^{bx}$,并估计$x=1.35$处的函数值。
\end{exercise}

\begin{solution}
先线性化处理,左右两边取对数:$\ln y=\ln a+bx$,然后$Y=\ln y$,$A =\ln a$,$Y = A + bx$。

解法方程组:$\begin{cases} nA+b\sum x_i=\sum Y_i \\ A\sum x_i+b\sum x_i^2=\sum x_iY_i & 
\end{cases}$

再求出真正的a,然后代入x的值即可求出估计函数值。
\end{solution}

\begin{remark}
线性化处理,法方程组。
\end{remark}

\begin{exercise}[Simpson公式数值求积]
$\begin{array} {ccccc}x_i & 1.0 & 1.8 & 2.6 \\ f(x_i) & 1 & -3 & 2 \end{array}$,用Simpson
公式计算$\int_{1.0}^{2.6}f(x)dx$
\end{exercise}

\begin{solution}
Simpson公式也被称为3点公式,适用于区间被分为偶数个等分(此处2等分,3节点)的情况。

Simpson公式的标准形式为:$I=\frac{h}{3}[f(x_0)+4f(x_1)+f(x_2)]$

此题中步长为0.8,其余值代入公式计算即可。
\end{solution}

\begin{remark}
记住Simpson公式即可。
\end{remark}

\begin{exercise}[梯形公式和Simpson公式求积]
$\begin{array} {cccc}x_i & 1.0 & 1.6 & 2.2 \\ f(x_i) & 0.85 & 0.59 & 0.44 \end{array}$,
分别用梯形公式和Simpson公式计算$\int_{1.0}^{2.2}\sin\frac{1}{x}dx$,并估计误差
\end{exercise}

\begin{solution}
步长0.6,复化梯形公式:$T_2=\frac{h}{2}[f(x_0)+2f(x_1)+f(x_2)]$,Simpson公式:
$S_2=\frac{h}{3}[f(x_0)+4f(x_1)+f(x_2)]$,复化梯形公式误差:$R_T=-\frac{(b-a)h^2}{12}f^
{\prime\prime}(\eta)$,Simpson公式误差:$R_S=-\frac{(b-a)h^4}{180}f^{(4)}(\eta)$
依次代入求解即可。
\end{solution}

\begin{remark}
记住梯形公式和Simpson公式。普通梯形公式为:$I\approx\frac{b-a}{2}[f(a)+f(b)]$,误差为:
$R_T=-\frac{(b-a)^3}{12}f^{\prime\prime}(\eta)$
\end{remark}

\begin{exercise}[复化梯形公式和Simpson公式求积]
$\begin{array} {cccccccccccccc}x_i & 1.0 & 1.2 & 1.4 & 1.6 & 1.8 & 2.0 & 2.2 & 2.4 & 2.6 
\\ f(x_i) & 1 & 2 & 0 & -1 & -3 & -1 & 1 & 3 & 2 \end{array}$,求$\int_{1.0}^{2.6}f(x)dx$
\end{exercise}

\begin{solution}
复化Simpson公式:$I_{n}(f)=\frac{h}{3}\left(f(a)+4\sum_{i=1}^{n-1}f(x_{i+1/2})+2\sum_{i=1}^
{n-1}f(x_{i})+f(b)\right)$,误差:$R_{n}(f)=-\frac{b-a}{180}h^{4}f^{(4)}(\xi)$
\end{solution}

\begin{remark}
记住复化梯形公式和Simpson公式即可。去掉头尾4倍的奇数项和2倍的偶数项和。
\end{remark}

\begin{exercise}[Gauss求积]
确定常数$A_i$,使求积公式:$\int_0^2f(x)dx\approx A_1f(0)+A_2f(1)+A_3f(2)$的代数精度尽可能高
,并求最高代数精度。该求积公式是否为Gauss型求积公式?
\end{exercise}

\begin{solution}
Gauss型:n个节点的插值型求积公式最高代数精度为2n-1,若达到此高度,则为Gauss型。本题节点数为3
若精度达到5则为Gauss型。

令$f(x) = x^k$ ($k=0, 1, 2, \dots$),代入公式左右两端,建立关于$A_i$的方程组:

$\begin{cases} & A_1+A_2+A_3=2 \\ & A_2+2A_3=2 \\ & A_2+4A_3=8/3 \end{cases}$

分别解得三个A,再依次代入$f(x)=x^3,f(x)=x^4,f(x)=x^5$等,直到第一次出现左右两端不成立。
\end{solution}

\begin{remark}
这种题就这样做
\end{remark}

\begin{exercise}[Gauss-Legendre求积和Gauss-Chebyshev求积]
用三点求积公式:$\int_0^{\frac{1}{3}}\frac{6x}{\sqrt{x\left(1-3x\right)}}dx$
\end{exercise}

\begin{solution}
将积分转为[-1,1]区间。$I=\int_{-1}^1\frac{t+1}{\frac{\sqrt{1-t^2}}{2\sqrt3}}
\cdot\frac{1}{6}dt=\int_{-1}^1\frac{2\sqrt3(t+1)}{6\sqrt{1-t^2}}dt=\frac{\sqrt3}{3}
\int_{-1}^1\frac{t+1}{\sqrt{1-t^2}}dt$

(1)使用Gauss-Chebyshev求积:$\int_{-1}^1\frac{g(t)}{\sqrt{1-t^2}}dt\approx\frac{\pi}{3}
\sum_{j=1}^3g(t_j)$,其中$g(t)=\frac{\sqrt{3}}{3}(t+1)$,$t_j=\cos\frac{(2j-1)\pi}{6}$
代入求解即可。

(2)使用Gauss-Legendre求积:令$F(t)=\frac{\sqrt{3}}{3}\cdot\frac{t+1}{\sqrt{1-t^2}}$,
选取标准节点,然后根据待定系数法求权重:$\int_{-1}^1f(x)dx\approx A_1f(x_1)+A_2f(x_2)+A_3f
(x_3)$,再代入公式即可得结果。
\end{solution}

\begin{remark}
$\int_{-1}^1\frac{f(x)}{\sqrt{1-x^2}}dx\approx\frac{\pi}{n}\sum_{j=1}^nf\left(\cos
\frac{2j-1}{2n}\pi\right)$

$\int_{-1}^1f(x)dx\approx\sum_{j=1}^nA_jf(x_j)$

标准节点,2点是根号1/3,3点是根号3/5和0
\end{remark}

\begin{exercise}[改进欧拉法]
用改进欧拉法求初值问题:$\begin{cases} \frac{dy}{dx}=x+y^2,\quad x>0 \\ y(0)=1 & 
\end{cases}$,解函数$y(x)$在$x=0.2$得近似值(取步长h=0.1)
\end{exercise}

\begin{solution}
预估步:利用当前点的斜率预估下一点的粗略值,$\bar{y}_{n+1}=y_n+hf(x_n,y_n)$

校正步:利用当前点和预估点的斜率平均值进行修正,$y_{n+1}=y_n+\frac{h}{2}[f(x_n,y_n)+f(x_{n+1},
\bar{y}_{n+1})]$

当前值:$x_0=0,y_0=1$,当前斜率:$f(x_0,y_0)=x_0+y_0^2=0+1=1$,预估步:$\bar{y_1}=y_0+
hf(x_0,y_0)=1+0.1\times1=1.1$,校正斜率:$f(x_1,\bar{y}_1)=1.31$,校正步:
$y_1=y_0+\frac{0.1}{2}[1+1.31]=1+0.05\times2.31=1+0.1155=1.1155$,进行第二步。

当前值:$x_1=0.1,y_1=1.1155$,当前斜率:$f(x_1,y_1)=0.1+(1.1155)^2\approx0.1+1.24434=
1.34434$,预估步:$\bar{y_2}=y_1+hf(x_1,y_1)=1.1155+0.1\times1.34434=1.24993$,
校正斜率:$f(x_2,\bar{y}_2)=1.76233$,校正步:
$y_2=y_1+\frac{0.1}{2}[1.34434+1.76233]=1.1155+0.05\times3.10667$
\end{solution}

\begin{remark}
当前值,代入算当前斜率,然后代入算预估值。用预估值算校正斜率,最后代入校正斜率和当前斜率的均值
算校正步。
\end{remark}

\begin{exercise}[常微分方程精度和稳定性问题]
$\begin{cases} y^{^{\prime}}(x)=f(x,y) \\ y(a)=y_0 & \end{cases}$,

(1)$y_{n+1}=y_{n}+hf(x_{n}+\frac{h}{2},y_{n}+\frac{h}{2}f(x_{n},y_{n}))$整体精度几阶?

(2)$y_{n+1}=y_{n}+hf(x_{n}+\frac{h}{4},y_{n}+\frac{h}{4}f(x_{n},y_{n}))$,$f(x,y)=
\lambda y$时的稳定性。若$\lambda=0$,求出步长$h$的区间使算法绝对稳定。
\end{exercise}

\begin{solution}
(1)题干公式为标准的二阶龙格-库塔格式(中点法)

$y(x_{n+1})=y(x_n)+hy^{\prime}(x_n)+\frac{h^2}{2!}y^{\prime\prime}(x_n)+\frac{h^3}{3!}
y^{\prime\prime\prime}(x_n)+O(h^4)$

$\begin{cases} & y^{\prime}(x_n)=f \\ & y^{\prime\prime}(x_n)=\frac{d}{dx}f(x,y)=
  f_x+f_yy^{\prime}=f_x+ff_y \end{cases}$

$y(x_{n+1})=y_n+hf+\frac{h^2}{2}(f_x+ff_y)+O(h^3)$

$f\left(x_n+\frac{h}{2},y_n+\frac{h}{2}f\right)=f+\left(f_x\cdot\frac{h}{2}+
f_y\cdot\frac{h}{2}f\right)+O(h^2)$

$y_{n+1}=y_n+hf+\frac{h^2}{2}(f_x+ff_y)+O(h^3)$

$R_{n+1}=y(x_{n+1})-y_{n+1}$,$R_{n+1}=O(h^3)$

(2)代入$f(x,y)=\lambda y$

$\begin{aligned} & k_1=\lambda y_n \\ & k_2=\lambda(y_n+\frac{h}{4}k_1)=
\lambda(y_n+\frac{h}{4}\lambda y_n)=\lambda(1+\frac{\lambda h}{4})y_n \\ 
& y_{n+1}=y_n+hk_2=y_n+h\lambda(1+\frac{\lambda h}{4})y_n \end{aligned}$

$y_{n+1}=\left(1+\lambda h+\frac{(\lambda h)^2}{4}\right)y_n$

令$z=\lambda h$,则放大因子:$E(z)=1+z+\frac{z^2}{4}$,绝对稳定要求$|E(z)|<1$,即:
$-1<1+z+\frac{z^2}{4}<1$,$-4<z<0$,$\frac{-4}{\lambda}>h>0$
\end{solution}

\begin{remark}
泰勒展开与迭代的应用
\end{remark}

\begin{exercise}[Jacobi迭代法]$\\x_1+2x_2-2x_3=1\\x_{1}+x_{2}+x_{3}=1\\2x_{1}+2x_{2}+x_{3}=1$

(1)系数矩阵A,求$\left\|A\right\|_1$和$cond_1(A)$

(2)$\\x_{1}+2x_{2}-2x_{3}=1.001\\x_{1}+x_{2}+x_{3}=0.998\\2x_{1}+2x_{2}+x_{3}=1$,求
相对误差:$\frac{\|\boldsymbol{\delta}x\|_{1}}{\|x\|_{1}}$

(3)写出Jacobi迭代格式和Gauss-Seidel迭代格式

(4)证明Jacobi迭代格式收敛。
\end{exercise}

\begin{solution}
(1)$A= \begin{pmatrix} 1 & 2 & -2 \\ 1 & 1 & 1 \\ 2 & 2 & 1 \end{pmatrix}$,
$\|A\|_1=5$(列模最大值),条件数$\text{cond}_1(A) = \|A\|_1 \cdot \|A^{-1}\|_1$

(2)误差估计公式为:$\frac{\|\delta x\|_1}{\|x\|_1}\leq\mathrm{cond}_1(A)\frac
{\|\delta b\|_1}{\|b\|_1}$

$b = [1, 1, 1]^T$,所以$\|b\|_1 = 3$,$\delta b = \tilde{b} - b = [0.001, -0.002, 0]^T$,
代入值即可计算出相对误差。

(3)Jacobi:$\begin{cases} x_1^{(k+1)}=1-2x_2^{(k)}+2x_3^{(k)} \\ x_2^{(k+1)}=1-x_1^
{(k)}-x_3^{(k)} \\ x_3^{(k+1)}=1-2x_1^{(k)}-2x_2^{(k)} & \end{cases}$,Gauss-Seidel:
$\begin{aligned} & \begin{cases} x_1^{(k+1)}=1-2x_2^{(k)}+2x_3^{(k)} \\ 
x_2^{(k+1)}=1-x_1^{(k+1)}-x_3^{(k)} \\ x_3^{(k+1)}=1-2x_1^{(k+1)}-2x_2^{(k+1)} & 
\end{cases} \end{aligned}$

(4)Jacobi迭代矩阵为:$B=-D^{-1}(L+U)$,D为A的对角阵,LU为上三角下三角阵,计算B的特征值
的最大值,找出谱半径即可,小于1说明收敛
\end{solution}

\begin{remark}
列模,条件数,相对误差,Jacobi迭代矩阵$A=D-L-U$,其中D为对角阵,LU为相反数下三角上三角阵。
Gauss-Seidel迭代矩阵。它们的收敛均要求迭代矩阵的谱半径(绝对值最大的特征值)小于1
\end{remark}

\begin{exercise}[牛顿迭代法]
$x=e^{-x}$,分析$x_{0}=0.5,x_{n+1}=e^{-x_{n}}\quad n=0,1,2,\cdots\cdots$的收敛性。
写出此方程的牛顿迭代格式,并问$x_0=0.5$,迭代是否收敛。
\end{exercise}

\begin{solution}
(1)$\phi(x) = e^{-x}$,$|\phi'(x)| < 1$,收敛

(2)$f(x) = x - e^{-x}$,$f'(x) = 1 + e^{-x}$

迭代公式:
$x_{n+1} = x_n - \frac{f(x_n)}{f'(x_n)} = x_n - \frac{x_n - e^{-x_n}}{1 + e^{-x_n}}$,
$x_{n+1} = \frac{(x_n + 1)e^{-x_n}}{1 + e^{-x_n}}$

只需要构造等式右边的关于x函数,求其导,证明导函数绝对值小于1,即可说明迭代格式收敛。
\end{solution}

\begin{remark}
掌握牛顿迭代法公式:$x_{n+1} = x_n - \frac{f(x_n)}{f'(x_n)}$
\end{remark}

\begin{exercise}[Newton迭代法]
$x^*$是$f(x)=0$的三重根,$f(x)$在$x^*$的邻域内有三阶连续导数。

(1)证明$f(x)=0$的Newton迭代法在$x^*$附近是线性收敛的。

(2)改变上面的Newton迭代法,使其在$x^*$附近有二阶收敛性。
\end{exercise}

\begin{solution}
Newton迭代法通用公式:$\psi(x) = x - \frac{f(x)}{f'(x)}$,
$f(x) = (x - x^*)^3 \cdot g(x), \quad \text{其中 } g(x^*) \neq 0$,
$f'(x) = 3(x - x^*)^2 g(x) + (x - x^*)^3 g'(x) = (x - x^*)^2 [3g(x) + (x - x^*)g'(x)]$,
$\psi^\prime(x)=1-\frac{[f^\prime(x)]^2-f(x)f^{\prime\prime}(x)}{[f^\prime(x)]^2}=
\frac{f(x)f^{\prime\prime}(x)}{[f^\prime(x)]^2}$,

若 $\psi'(x^*) \neq 0$,则为线性收敛,且收敛比为 $|\psi'(x^*)|$

若 $\psi'(x^*) = 0$ 且 $\psi''(x^*) \neq 0$,则为二阶收敛

对于 $m$ 重根,Newton 法的导数值满足:$\psi'(x^*) = 1 - \frac{1}{m}=2/3$,是线性收敛。

(2)$\phi(x) = x - 3 \frac{f(x)}{f'(x)}$,导数为0具有二阶收敛性
\end{solution}

\begin{remark}
Newton法的线性收敛以及二阶收敛问题。
\end{remark}

\subsection{重难点学习}
重点复习方向:

1.插值法:Lagrange插值,Newton插值,等距节点的Newton插值,Hermite插值,龙格现象:

Lagrange插值:$L_n(x)$与$l_n(x)$之间的关系,以及插值基函数的求解方式,截断误差。

Newton插值:主要关注差商表的构造方法。Hermite插值:Newton插值的推广,带导数的,误差为:
$R_{2n+1}(x)=f(x)-H_{2n+1}(x)=\frac{f^{(2n+2)}(\xi)}{(2n+2)!}\omega_{n+1}^{2}(x)$。

龙格现象:指的是对于某些函数,使用等距节点构造高阶插值多项式时,在插值区间的边缘的误差可能很大的现象
克服方法:chebyshev节点,分段低阶插值函数(分段线性函数,分段Hermite, 分段spline函数等)

2.带权内积,最佳二次平方逼近

带权内积:$(f,g)$称为函数f和g在[a,b]上的以权为$\rho$的带权内积。
$(f,g)=\int_a^b\rho(x)f(x)g(x)dx$

最佳二次逼近:找多项式,用法方程组算系数,使误差的绝对平方和最小

3.正交多项式

带权内积为0,除了两个相同的多项式做带权内积不为零。

4.曲线拟合的最小二乘法,正交多项式的最佳二次逼近

最佳二次逼近:找法方程组,或者用勒让德,切比雪夫正交多项式:$1,x,1/2(3x^2-1),1/2(5x^3-3x)$
和$1,x,2x^2-1,4x^3-3x$

5.梯形,Simpson积分公式

梯形积分公式:$I_n(f)=\frac{b-a}{2}(f(a)+f(b))$,误差:$R_T=-\frac{(b-a)^3}{12}f^
{\prime\prime}(\eta)$,代数精度为2。

复化梯形公式:$T_2=\frac{h}{2}[f(x_0)+2f(x_1)+f(x_2)]$,误差:$R_T=-\frac{(b-a)h^2}{12}f^
{\prime\prime}(\eta)$,代数精度为2。

Simpson公式:$I_n(f)=\frac{b-a}{6}(f(a)+4f(\frac{a+b}{2})+f(b))$,误差:$R_S=-\frac{(b-a)
h^4}{180}f^{(4)}(\eta)$,代数精度为3。

复化Simpson公式:$I_{n}(f)=\frac{h}{3}\left(f(a)+4\sum_{i=1}^{n-1}f(x_{i+1/2})+2\sum_{i=1}^
{n-1}f(x_{i})+f(b)\right)$,误差:$R_{n}(f)=-\frac{b-a}{180}h^{4}f^{(4)}(\xi)$,代数精度为3

6.插值型求积公式

3阶可用Simpson公式求,或者待定系数法求。3阶的含义为:对$1,x,x^2,x^3$均能精确成立。

7.求积公式的收敛性与稳定性

对于复化梯形公式和复化 Simpson 公式,只要被积函数 $f(x)$ 在区间 $[a, b]$ 上连续,
其收敛性就是有保证的,一个求积公式稳定的充分条件是:其求积系数 $A_i$ 全部为正

8.Newton-Cotes公式

n=2k时,Newton-Cotes公式至少有2k+1阶代数精度。
系数:$\frac{7}{90},\frac{32}{90},\frac{12}{90},\frac{32}{90},\frac{7}{90}$
,$I(f)\approx(b-a)\sum_{k=0}^nC_k^{(n)}f(x_k)$

9.Gauss型求积公式

Gauss-Legendre,Gauss-Chebyshev求积公式。区间有限制,公式也有限制,选择的点也有限制。

$C=\frac{\int_a^b\rho(x)x^{m+1}dx-\sum A_ix_i^{m+1}}{(m+1)!}$求误差的通式。m是精度

10.Euler格式求ODE,改进的Euler

迭代的应用。

11.ODE(常微分方程)数值求解的误差与计算精度

局部截断误差的幂次比精度p大1。

12.Runge-Kutta方法

二阶RK方法公式:$\begin{cases} k_1=f(x_n,y_n) \\ k_2=f(x_n+h,y_n+hk_1) \\ y_{n+1}=
y_n+\frac{h}{2}(k_1+k_2) & \end{cases}$

四阶RK方法公式:$\begin{cases} k_1=f(x_n,y_n) \\ k_2=f(x_n+\frac{h}{2},y_n+\frac{h}{2}k_1) 
\\ k_3=f(x_n+\frac{h}{2},y_n+\frac{h}{2}k_2) \\ k_4=f(x_n+h,y_n+hk_3) \\ 
y_{n+1}=y_n+\frac{h}{6}(k_1+2k_2+2k_3+k_4) & \end{cases}$

核心就是把k算出来

13.ODE单步法的收敛性与稳定性

收敛性类似于误差为0,稳定性要求放大因子小于1。放大因子的是导数和原函数之间的倍数$\lambda$

14.误差分析,条件数

行范数:$\|A\|_{\infty}$,列范数:$\left\|A\right\|_{1}$,二范数:$\|A\|_{2}=\sqrt
{\lambda_{max}(A^{T}A)}$等于A的最大奇异值,F范数:$\|A\|_F=\sqrt{\sum_{i,j=1}^na_{ij}^2}$
所有元素平方和开根号。

误差分析:$\|A^{-1}\|\|A\|$称为条件数$cond(A)$

A有舍入误差的话$\frac{\|\delta x\|}{\|x\|}\leq\frac{cond(A)\|\delta A\|/\|A\|}
{1-cond(A)\|\delta A\|/\|A\|}$

b有舍入误差的话:$\frac{\|\delta x\|}{\|x\|}\leq cond(A)\frac{\|\delta b\|}{\|b\|}$

Gauss消元法,就是行化简为上三角。LU分解,和前面一样,但是L的主对角线为1。Cholesky分解:$A=LDL^T$

15.迭代法:Jacobi迭代,Gauss-Seidel迭代,迭代法的收敛分析

Jacobi迭代:$A=L+D+U$,$x=-D^{-1}(L+U)x+D^{-1}b$,$x^{(k+1)}=-D^{-1}(L+U)x^{(k)}+D^{-1}b$,

Gauss-Seidel迭代:$A=L+D+U$,$Dx=b-Lx-Ux$,$Dx^{(k+1)}=b-Lx^{(k+1)}-Ux^{(k)}$,
$x^{(k+1)}=(D+L)^{-1}(b-Ux^{(k)})$

迭代法收敛性:谱半径小于1,就是迭代矩阵的最大特征值绝对值小于1。

16.非线性方程迭代求根;p阶收敛,Newton迭代法;

构造原则:尽量让$x$出现在分母上或开方符号内,这样通常能减小导数的绝对值,要求导数的绝对值小于1
如果 $\phi'(x^*) \neq 0$,则为线性收敛。 $|\phi'(x^*)|$ 越接近 0,收敛越快;
如果 $\phi'(x^*) = 0$ 且 $\phi''(x^*) \neq 0$,则为平方收敛(如牛顿法)。

一般选用不动点迭代(需要构造,要求导数小于1),或者牛顿迭代法,直接构造记住公式即可。

$x_{k+1}=x_k-\frac{f(x_k)}{f^{\prime}(x_k)}$

p阶收敛:$\lim_{k\to\infty}\frac{e_{k+1}}{e_{k}^{p}}=M\neq0$

Newton下山法:$x_{n+1}=x_n-\lambda\frac{f(x_n)}{f^{\prime}(x_n)}$,判断是否有:
$|f(x_{n+1})|<|f(x_n)|$,如果没有减半$\lambda$继续尝试直到满足为止。下一步重新从1开始

Newton截弦法:$x_{n+1}=x_n-f(x_n)\cdot\frac{x_n-x_{n-1}}{f(x_n)-f(x_{n-1})}$

非线性Newton法:构造雅可比矩阵,$\mathbf{J}(x_1,x_2)= \begin{bmatrix} \frac{\partial f_1}
{\partial x_1} & \frac{\partial f_1}{\partial x_2} \\ \frac{\partial f_2}{\partial x_1} 
& \frac{\partial f_2}{\partial x_2} \end{bmatrix}$。

$\mathbf{J}(\mathbf{x}^{(k)})\Delta\mathbf{x}=-\mathbf{F}(\mathbf{x}^{(k)})$,其中F为原
非线性方程组代入具体值构成的矩阵,J为代入具体值矩阵,可以解出增量x,然后将增量x加入到原x中,接下来
继续迭代直到满足精度即可。

\section{数理统计}

\subsection{作业学习}

\begin{exercise}[泊松分布,联合概率密度,期望方差样本方差]
总体X服从泊松分布$\pi(\lambda)$,$X_i(\mathrm{i=1,2,\ldots,n})$为来自总体X的样本。

(1)$(X_1,X_2,...,X_n)^T$的联合分布概率密度函数;

(2)$E(\bar{X}),D(\bar{X}),E(\mathrm{S}^{2}),E(\frac{1}{n}\sum_{i=1}^{n}(X_{i}-\bar{X})^
{2})$
\end{exercise}

\begin{solution}
(1)泊松分布公式:$P(X=k)=\frac{\lambda^ke^{-\lambda}}{k!}$,$P(X_1=x_1,X_2=x_2,\ldots,X_n=x_n)
=\prod_{i=1}^n\frac{\lambda^{x_i}e^{-\lambda}}{x_i!}=\frac{\lambda^{\sum_{i=1}^nx_i}e^
{-n\lambda}}{\prod_{i=1}^nx_i!}$。

(2)$\begin{aligned} & \bar{X}=\frac{1}{n}\sum_{i=1}^nX_i \\ & E(\bar{X})=E
(\frac{1}{n}\sum X_i)=\frac{1}{n}\sum E(X_i)=\frac{1}{n}(n\lambda)=\lambda \end{aligned}$

$D(\bar{X})=D(\frac{1}{n}\sum X_i)=\frac{1}{n^2}\sum D(X_i)=\frac{1}{n^2}(n\lambda)=
\frac{\lambda}{n}$

样本$S^2$的公式为:$S^2=\frac{1}{n-1}\sum_{i=1}^n(X_i-\bar{X})^2$,样本方差 $S^2$ 都是总体
方差 $\sigma^2$ 的无偏估计。所以$E(S^2) = \lambda$,也可以直接计算。

$E(\frac{1}{n}\sum(X_i-\bar{X})^2)=\frac{n-1}{n}E(S^2)=\frac{\mathbf{n-1}}{\mathbf{n}}
\lambda$
\end{solution}

\begin{remark}
泊松分布公式记住,以及其期望和方差都是$\lambda$。样本公式也要记住。样本均值为无偏估计,
样本方差为$\frac{2\sigma^4}{n-1}$
\end{remark}

\begin{exercise}[经验分布函数]
$(3,2,3,4,2,3,5,7,9,3)^{T}$为来自总体X的样本,求经验分布函数$F_{10}(x)$
\end{exercise}

\begin{solution}
按大小排布同时统计出现次数即可。

$\begin{gathered} F_{10}(x)= \begin{cases} 0, & x<2 \\ 0.2, & 2\leq x<3 \\ 0.6, & 
3\leq x<4 \\ 0.7, & 4\leq x<5 \\ 0.8, & 5\leq x<7 \\ 0.9, & 7\leq x<9 \\ 1.0, & 
x\geq9 \end{cases} \end{gathered}$
\end{solution}

\begin{remark}
经验分布函数求解思路。
\end{remark}

\begin{exercise}[分布函数相关问题]
连续随机变量X的密度函数$p(x)$是一个偶函数,$F(x)$为其分布函数,求证对于任意$a>0$,有

$\begin{aligned} & (1).F(-a)=1-F(a)=0.5-\int_{0}^{a}p(x)dx; \\ & (2).P(|X|<a)=2F(a)-1; 
\\ & (3).P(|X|>a)=2[1-F(a)]. \end{aligned}$

(4)由正态总体N(100,4)抽取两组独立样本$X_{1},X_{2},...,X_{10}$和$Y_{1},Y_{2},...,Y_{40}$,
样本均值分别为${\bar{X}}$和${\bar{Y}}$,求$P(|\bar{X}-\bar{Y}|>0.2)$
\end{exercise}

\begin{solution}
(1)$F(-a) = \int_{-\infty}^{-a} p(x) dx$,同时$\int_a^{+\infty} p(x) dx = 1 - \int_
{-\infty}^a p(x) dx = 1 - F(a)$,$F(-a) = \int_{-\infty}^{-a} p(x) dx = \int_{-\infty}^0 
p(x) dx - \int_{-a}^0 p(x) dx$,$\int_{-a}^0 p(x) dx = \int_0^a p(x) dx$。得证

(2)$P(|X| < a) = P(-a < X < a) = F(a) - F(-a)$,代入(1)中结论。

(3)$P(|X| > a) = 1 - P(|X| < a)$,代入(2)中结论。

(4)$\bar{X} \sim N(\mu, \frac{\sigma^2}{n_1}) = N(100, \frac{4}{10}) = N(100, 0.4)$,
$\bar{Y} \sim N(\mu, \frac{\sigma^2}{n_2}) = N(100, \frac{4}{40}) = N(100, 0.1)$,
$E(\bar{X} - \bar{Y}) = 100 - 100 = 0$,
$D(\bar{X} - \bar{Y}) = D(\bar{X}) + D(\bar{Y}) = 0.4 + 0.1 = 0.5$,
两组样本独立同分布,差值仍服从正态分布:$Z_{diff} = \bar{X} - \bar{Y} \sim N(0, 0.5)$。

$P\left(\left|\frac{\bar{X} - \bar{Y} - 0}{\sqrt{0.5}}\right| > \frac{0.2}{\sqrt{0.5}}
\right) = P(|Z| > \frac{0.2}{0.707}) \approx P(|Z| > 0.283)$,
$P(|Z| > 0.283) = 2[1 - \Phi(0.283)]$查表即可得结果。
\end{solution}

\begin{remark}
记住分布函数的性质即可。
\end{remark}

\begin{exercise}[F分布性质]
$X_1,X_2$来自正态分布$N(0,\sigma^{2})$的独立样本,求$(\frac{X_{1}+X_{2}}{X_{1}-X_{2}})^{2}$
的分布
\end{exercise}

\begin{solution}
$U = X_1 + X_2$,$V = X_1 - X_2$。

$E(U) = E(X_1) + E(X_2) = 0 + 0 = 0$,$D(U) = D(X_1) + D(X_2) = \sigma^2 + \sigma^2 = 
2\sigma^2$;$E(V) = E(X_1) - E(X_2) = 0 - 0 = 0$,$D(V) = D(X_1) + D(X_2) = \sigma^2 + 
\sigma^2 = 2\sigma^2$;$U \sim N(0, 2\sigma^2)$,$V \sim N(0, 2\sigma^2)$。

对于正态分布,不相关性等价于独立性,计算协方差:$Cov(U, V) = Cov(X_1 + X_2, X_1 - X_2)
= Cov(X_1, X_1) - Cov(X_1, X_2) + Cov(X_2, X_1) - Cov(X_2, X_2)
= D(X_1) - D(X_2) = \sigma^2 - \sigma^2 = 0$,协方差为0,且$(U,V)$服从联合正态分布,所以U和V
独立。

$\frac{U}{\sqrt{2\sigma^2}}\sim N(0,1)\frac{V}{\sqrt{2\sigma^2}}\sim N(0,1)$
$Y = \left( \frac{U}{V} \right)^2 = \frac{U^2}{V^2} = \frac{\left( \frac{U}
{\sqrt{2\sigma^2}} \right)^2}{\left( \frac{V}{\sqrt{2\sigma^2}} \right)^2}$
,于是$Y \sim F(1, 1)$。
\end{solution}

\begin{remark}
卡方分布:$Z_1, \dots, Z_k$是k个相互独立的标准正态分布随机变量,$\sum Z_i^2$服从自由度为 $k$ 
的卡方分布记作 $\chi^2(k)$,卡方分布的期望为 $k$,方差为 $2k$。
以及恒有公式$\frac{(n-1)S_n^2}{\sigma^2}\sim\chi^2(n-1)$

若两个随机变量 $U \sim \chi^2(d_1)$ 和 $V \sim \chi^2(d_2)$ 且相互独立,则它们的比值
(经过自由度归一化后) $F = \frac{U/d_1}{V/d_2}$ 服从自由度为 $(d_1, d_2)$ 的 $F$ 分布
\end{remark}

\begin{exercise}[t分布性质]
$X_{1},X_{2},\ldots,X_{n+1}$来自正态总体$N(\mu,\sigma^{2})$的独立样本,$\overline{X_{n}}=
\frac{1}{n}\sum_{i=1}^{n}X_{i}$和$S_n{}^2=\frac{1}{n-1}\sum_{i=1}^n(X_i-\bar{X})^2$,
求常数c使得$\mathrm{c}\frac{X_{n+1}-\overline{X_{n}}}{S_{n}}$服从t分布。
\end{exercise}

\begin{solution}
t分布的构造:$Z \sim N(0, 1)$,$Y \sim \chi^2(k)$,$Z$ 与 $Y$ 相互独立,
$T = \frac{Z}{\sqrt{Y/k}} \sim t(k)$,其中k为自由度

$D = X_{n+1} - \overline{X}_n$,$E(D) = E(X_{n+1}) - E(\overline{X}_n) = \mu - \mu = 0$,
$Var(D) = Var(X_{n+1}) + Var(\overline{X}_n) = \sigma^2 + \frac{\sigma^2}{n} = \sigma^2 
\left( \frac{n+1}{n} \right)$。

标准化:$Z = \frac{X_{n+1} - \overline{X}_n}{\sigma \sqrt{\frac{n+1}{n}}} \sim N(0, 1)$

$S_n^2 = \frac{1}{n-1} \sum_{i=1}^n (X_i - \overline{X}_n)^2$,恒有$\frac{(n-1)S_n^2}
{\sigma^2}\sim\chi^2(n-1)$,令$Y = \frac{(n-1)S_n^2}{\sigma^2}$,自由度 $k = n-1$。

$T = \frac{\frac{X_{n+1} - \overline{X}_n}{\sigma \sqrt{\frac{n+1}{n}}}}{\sqrt
{\frac{(n-1)S_n^2 / \sigma^2}{n-1}}} = \frac{X_{n+1} - \overline{X}_n}{\sigma \sqrt
{\frac{n+1}{n}}} \cdot \frac{1}{S_n / \sigma} = \sqrt{\frac{n}{n+1}} \cdot \frac
{X_{n+1} - \overline{X}_n}{S_n} \sim t(n-1)$于是可知$c = \sqrt{\frac{n}{n+1}}$
\end{solution}

\begin{remark}
t分布的结构记住。
\end{remark}

\begin{exercise}[离散的矩估计与最大似然估计]
总体X服从几何分布,$P(X=k)=p(1-p)^{k-1}.k=1.2...$,而$(X_{1},X_{2},...,X_{n})^{T}$为总体
的样本,求p的矩估计与最大似然估计。
\end{exercise}

\begin{solution}
几何分布的总体期望均值为:$E(X) = \frac{1}{p}$

总体均值 $E(X)$ 等于样本均值 $\bar{X}$,$\frac{1}{p}=\bar{X}$,解出$\hat{p}_{MME}=\frac{1}
{\bar{X}}$

样本$X_i$独立同分布,似然函数为各概率之积:$L(p)=\prod_{i=1}^nP(X=X_i)=\prod_{i=1}^np(1-p)^
{X_i-1}=p^n(1-p)^{\sum_{i=1}^nX_i-n}$

取对数似然函数:$\ln L(p)=n\ln p+(\sum_{i=1}^nX_i-n)\ln(1-p)$

求导并令导数为0:$\frac{d\ln L(p)}{dp}=\frac{n}{p}-\frac{\sum_{i=1}^nX_i-n}{1-p}=0$

解得:$\hat{p}_{MLE}=\frac{1}{\bar{X}}$
\end{solution}

\begin{remark}
记住几何分布公式,以及其期望为$E(X) = \frac{1}{p}$,方差为$D(X) = \frac{1-p}{p^2}$,矩估计和
最大似然估计。
\end{remark}

\begin{exercise}[连续的矩估计与最大似然估计]
总体的分布密度函数为:$f(x)=\left\{ \begin{array} {cc}(\theta+1)x^\theta, & \quad0<x<1 \\ 
0, & \quad\text{其他} \end{array}\right.$

其中$\theta>-1$,$(X_1,X_2,...,X_n)^T$为其样本,求$\theta$的矩估计与最大似然估计。当样本值为:
$(0.1,0.2,0.9,0.8,0.7,0.7)^{T}$求$\theta$的估计值。
\end{exercise}

\begin{solution}
$E(X) = \int_{0}^{1} x \cdot f(x) dx = \int_{0}^{1} x \cdot (\theta + 1)x^{\theta} dx$,
$E(X) = (\theta + 1) \int_{0}^{1} x^{\theta+1} dx = (\theta + 1) \left[ \frac
{x^{\theta+2}}{\theta + 2} \right]_{0}^{1} = \frac{\theta + 1}{\theta + 2}$

$\frac{\theta+1}{\theta+2}=\bar{X}$,解得$\hat{\theta}_{MME}=\frac{2\bar{X}-1}{1-\bar{X}}$

$L(\theta)=\prod_{i=1}^n(\theta+1)x_i^\theta=(\theta+1)^n\left(\prod_{i=1}^nx_i\right)^
\theta$,$\ln L(\theta)=n\ln(\theta+1)+\theta\sum_{i=1}^n\ln x_i$,$\frac{d\ln L(\theta)}
{d\theta}=\frac{n}{\theta+1}+\sum_{i=1}^n\ln x_i=0$,解得:$\hat{\theta}_{MLE}=-\frac{n}
{\sum_{i=1}^n\ln x_i}-1$

代入具体数值:样本均值为$\bar{X} = \frac{0.1 + 0.2 + 0.9 + 0.8 + 0.7 + 0.7}{6} = 
\frac{3.4}{6} \approx 0.5667$,$\hat{\theta}_{MME} = \frac{2(0.5667) - 1}{1 - 0.5667} = 
\frac{1.1334 - 1}{0.4333} \approx \mathbf{0.3079}$,$\sum \ln x_i$:
$\ln(0.1) + \ln(0.2) + \ln(0.9) + \ln(0.8) + \ln(0.7) + \ln(0.7)$
$\approx -2.3026 - 1.6094 - 0.1054 - 0.2231 - 0.3567 - 0.3567 \approx -4.9539$,
$\hat{\theta}_{MLE} = -\frac{6}{-4.9539} - 1 \approx 1.2112 - 1 = \mathbf{0.2112}$
\end{solution}

\begin{remark}
记住连续密度函数怎么求总体均值(期望)。
\end{remark}

\begin{exercise}[有效估计]
X服从正态分布$N(0,\sigma^{2})$,$(X_{1},X_{2},...,X_{n})^{T}$为其样本,说明$\widehat
{\sigma^{2}}=\frac{1}{n}\sum_{i=1}^{n}{X_{i}}^{2}$是$\sigma^2$的有效估计,而
$\widehat{\sigma^{2}}=\frac{1}{n-1}\sum_{i=1}^{n}(X_{i}-\bar{X})^{2}$不是。
\end{exercise}

\begin{solution}
有效估计的判定:必须无偏,期望等于参数,同时它在所有无偏估计量中,方差达到理论最小值。

$X_i \sim N(0, \sigma^2)$,可知$E(X_i^2) = \sigma^2$。$E(\widehat{\sigma}_1^2) = \frac{1}
{n} \sum_{i=1}^n E(X_i^2) = \frac{1}{n} \cdot n\sigma^2 = \sigma^2$,因此$\widehat{\sigma}
_1^2$ 是 $\sigma^2$ 的无偏估计

又$X_i^2 / \sigma^2 \sim \chi^2(1)$,则$D(X_i^2) = 2\sigma^4$,$D(\widehat{\sigma}_1^2) 
= \frac{1}{n^2} \sum_{i=1}^n D(X_i^2) = \frac{1}{n^2} \cdot 2n\sigma^4 = \mathbf
{\frac{2\sigma^4}{n}}$

$\widehat{\sigma}_2^2 = \frac{1}{n-1} \sum_{i=1}^n (X_i - \bar{X})^2$为样本,所以一定是
无偏估计。样本方差为:$D(\widehat{\sigma}_2^2) = \frac{2\sigma^4}{n-1}$,所以$D(\widehat
{\sigma}_2^2) > D(\widehat{\sigma}_1^2)$

理论最小方差判断:写出单个观测值 $X$ 的概率密度函数,$f(x;\sigma^2)=\frac{1}{\sqrt
{2\pi\sigma^2}}e^{-\frac{x^2}{2\sigma^2}}$,然后取ln,再对$\sigma^2$求二次导,然后对二次导
求负期望得到$I(\sigma^2)=-E\left[\frac{1}{2(\sigma^2)^2}-\frac{x^2}{(\sigma^2)^3}\right]=
-\left(\frac{1}{2(\sigma^2)^2}-\frac{\sigma^2}{(\sigma^2)^3}\right)=\frac{1}{2\sigma^4}$
,于是下界公式:$D(\widehat{\sigma}^2)\geq\frac{1}{n\cdot I(\sigma^2)}=\frac{1}
{n\cdot\frac{1}{2\sigma^4}}=\frac{2\sigma^4}{\mathbf{n}}$
\end{solution}

\begin{remark}
无偏估计与有效估计的概念。
\end{remark}

\begin{exercise}[无偏估计问题]
$(X_{1},X_{2},...,X_{n})^{T}$是总体X的样本,常数$a_{i}>0,(\mathrm{i=1,2,\ldots,n})$,且
$\sum_{i=1}^{n}a_{i}=1$。

(1)$\sum_{i=1}^{n}a_{i}X_{i}$是$E(X)$的无偏估计

(2)在所有$\sum_{i=1}^{n}a_{i}X_{i}$中,$\bar{X}$是方差最小的$E(X)$无偏估计
\end{exercise}

\begin{solution}
(1)设总体的期望(均值)为 $E(X) = \mu$,$E\left( \sum_{i=1}^n a_i X_i \right) = 
\sum_{i=1}^n a_i E(X_i)$,由于 $X_i$ 是来自总体 $X$ 的样本,因此每个 $E(X_i) = \mu$,
$E\left( \sum_{i=1}^n a_i X_i \right) = \sum_{i=1}^n a_i \mu = \mu \left( \sum_{i=1}
^n a_i \right) = \mu$

(2)样本均值 $\bar{X}$ 对应于 $a_1 = a_2 = \dots = a_n = \frac{1}{n}$ 的特殊情况

$D\left( \sum_{i=1}^n a_i X_i \right) = \sum_{i=1}^n a_i^2 D(X_i) = \sigma^2 \sum_
{i=1}^n a_i^2$

我们要寻找一组 $a_i$ 使得 $\sum a_i^2$ 最小,且满足约束条件 $\sum a_i = 1$,
设 $L = \sum a_i^2 - \lambda (\sum a_i - 1)$。对每个 $a_i$ 求偏导并令其为 0,
$\frac{\partial L}{\partial a_i} = 2a_i - \lambda = 0 \implies a_i = \frac{\lambda}{2}$
这意味着所有的 $a_i$ 必须相等。由 $\sum a_i = 1$ 可得:$n \cdot a_i = 1 \implies a_i = 
\frac{1}{n}$,当 $a_i = \frac{1}{n}$ 时,估计量恰好是 $\bar{X} = \frac{1}{n} \sum X_i$。
此时方差最小
\end{solution}

\begin{remark}
拉格朗日乘数法寻找极值解问题。
\end{remark}

\begin{exercise}[置信区间]
对铅的比重进行 16 次测量,得到 16 个测量值的平均值为 2.705,而样本方差为$0.029^2$,
假定测量结果$X$服从正态分布,试求铅的平均比重的置信度为$95\%$的置信区间.
\end{exercise}

\begin{solution}
由题可知信息:样本量:$n=16$,样本均值:$\bar{x}=2.705$,样本标准差:$s=0.029$,置信度:
$1-\alpha=0.95$,$\alpha=0.05$,总体分布$X \sim N(\mu, \sigma^2)$,其中$\sigma^2$未知。

当总体方差未知且样本量较小时,样本均值标准化的统计量服从自由度为 $n-1$ 的$t$分布:
$T = \frac{\bar{X} - \mu}{S / \sqrt{n}} \sim t(n-1)$,需要查找自由度为 $16 - 1 = 15$ 的 
$t$ 分布在双侧概率为 $0.05$ 时的分位数,$t_{\alpha/2}(n-1) = t_{0.025}(15) \approx 
\mathbf{2.131}$

置信区间的计算公式为:$\left[ \bar{x} - t_{\alpha/2}(n-1) \frac{s}{\sqrt{n}}, 
\quad \bar{x} + t_{\alpha/2}(n-1) \frac{s}{\sqrt{n}} \right]$,代入值计算即可得结果。
\end{solution}

\begin{remark}
记忆置信区间计算公式。方差未知利用样本方差代替,采用t分布进行估计。均值问题用t检验
\end{remark}

\begin{exercise}[假设检验]
测定溶液中某种物质的水平,假设其服从正态分布$N(\mu,\sigma^2)$,它的 10 个测
定值给出$\bar{x}=0.452\%,S=0.037\%$,试在水平$\alpha=0.05$下检验假设:
(1). $H_{0}\nobreak {: } \mu \geq 0. 5\%  \Longleftrightarrow  H_{1}\nobreak {: } 
\mu < 0. 5\%$ ;
(2) . $H_{0}\nobreak {: }$ $\sigma \geq 0. 04\%$ $\Leftrightarrow$ $H_{1}\nobreak {: }$ 
$\sigma < 0. 04\%$ .
\end{exercise}

\begin{solution}
由题可知信息:样本量:$n=10$,样本均值:$\bar{x}=0.452\%$,样本标准差:$s=0.037\%$,
显著性水平$\alpha=0.05$,总体分布$X \sim N(\mu, \sigma^2)$,其中$\mu,\sigma^2$未知。

(1)$H_0: \mu \ge 0.5\% \iff H_1: \mu < 0.5\%$均值检验。总体方差 $\sigma^2$ 未知,使用 $t$ 
检验$t = \frac{\bar{x} - \mu_0}{S / \sqrt{n}} = \frac{0.452\% - 0.5\%}{0.037\% / 
\sqrt{10}} \approx \frac{-0.048}{0.0117} \approx \mathbf{-4.103}$。

这是一个左侧检验。查 $t$ 分布表,自由度 $df = 10 - 1 = 9$:
$t_{\alpha}(9) = t_{0.05}(9) = 1.833$。拒绝域为 $t < -1.833$

因为观察值 $t = -4.103 < -1.833$,落在拒绝域内。
结论:在 $\alpha=0.05$ 水平下,拒绝 $H_0$,接受 $H_1$。即认为该物质水平显著低于 $0.5\%$

(2)对总体方差(或标准差)的检验使用 $\chi^2$ 检验,$\chi^2 = \frac{(n-1)S^2}{\sigma_0^2} = 
\frac{9 \times (0.037\%)^2}{(0.04\%)^2} = \frac{9 \times 0.001369}{0.0016} \approx 
\mathbf{7.701}$,这是一个左侧检验。查 $\chi^2$ 分布表,自由度 $df = 9$:
$\chi^2_{1-\alpha}(9) = \chi^2_{0.95}(9) = 3.325$。拒绝域为 $\chi^2 < 3.325$

因为观察值 $\chi^2 = 7.701 > 3.325$,没有落在拒绝域内。
结论:在 $\alpha=0.05$ 水平下,不能拒绝 $H_0$。即没有充足证据认为标准差显著小于 $0.04\%$
\end{solution}

\begin{remark}
分位点的选取有讲究,均值检验t分布,方差检验卡方分布。t是对称拒绝域,卡方是单侧拒绝域。
t要找0.95分位点可以化为找0.05分位点加一个负号。卡方必须要找0.95分位点。
\end{remark}

\begin{exercise}[假设检验]
9名运动员在初进学校时接受体育训练的检测,经过一个星期的训练后再进行检测,检测结果记分如下:
入学初$X:76,71,57,49,70,69,26,65,59;$ 训练后$Y:81,85,52,52,70,63,33,83,62;$ 假定分数服从正态
分布,请在显著性水平0.05下判断判断运动员在训练后是否有进步?
\end{exercise}

\begin{solution}
首先计算每位运动员训练前后的分差 $D_i = Y_i - X_i$,样本量:$n = 9$,差值均值 $\bar{d}$:
$\bar{d} = \frac{5 + 14 - 5 + 3 + 0 - 6 + 7 + 18 + 3}{9} = \frac{39}{9} \approx 
\mathbf{4.333}$,$s_d = \sqrt{\frac{\sum D_i^2 - n\bar{d}^2}{n-1}} = \sqrt
{\frac{701 - 9 \times (4.333)^2}{8}} = \sqrt{\frac{532}{8}} \approx \mathbf{8.155}$

原假设 $H_0$:$\mu_D \le 0$(训练后没有进步或退步了),
备择假设 $H_1$:$\mu_D > 0$(训练后有显著进步)这是一个右侧单侧检验,显著性水平 $\alpha = 0.05$。

构造统计量:$t = \frac{\bar{d} - 0}{s_d / \sqrt{n}} = \frac{4.333}{8.155 / \sqrt{9}} = 
\frac{4.333}{2.718} \approx \mathbf{1.594}$

自由度 $df = n - 1 = 8$。查 $t$ 分布表,单侧 $\alpha = 0.05$ 下:
$t_{0.05}(8) = \mathbf{1.860}$,拒绝域为 $t > 1.860$。

因为观察值 $t = 1.594 < 1.860$,统计量未落在拒绝域内。

在 $\alpha = 0.05$ 的显著性水平下,不能拒绝原假设 $H_0$。
这意味着,虽然从数据上看平均分有所提高(增加了 4.333 分),但这种差异在统计学上并不显著。因此,
不能认为运动员在训练后有显著进步
\end{solution}

\begin{remark}
样本标准差$S = \sqrt{\frac{\sum_{i=1}^{n} (X_i - \bar{X})^2}{n-1}}$

总体标准差$\sigma = \sqrt{\frac{\sum_{i=1}^{n} (X_i - \mu)^2}{n}}$

左侧检验为原假设大于等于,右侧检验为原假设小于等于。t分布左侧检验查$\alpha$带负号,右侧检验
查$\alpha$不用带负号。卡方分布左侧检验查$1-\alpha$,右侧检验查$\alpha$。
\end{remark}

\begin{exercise}[0-1分布大样本假设检验]
如果一批产品的废品率不超过 0.02,这批产品即可被接收。现在随机抽取480 件产品检查后发现 12 件废品,
请问在$\alpha=0.05$水平下这批废品是否可以被接受?
\end{exercise}

\begin{solution}
原假设 $H_0$:$p \le 0.02$(废品率未超标,可以接收),
备择假设 $H_1$:$p > 0.02$(废品率显著超标,不能接收)。

总体比例限值:$p_0 = 0.02$样本量:$n = 480$样本中的废品数:$x = 12$样本比例:
$\hat{p} = \frac{12}{480} = 0.025$显著性水平:$\alpha = 0.05$

在大样本条件下,样本比例 $\hat{p}$ 近似服从正态分布。检验统计量 $Z$ 的计算公式为:
$Z=\frac{\hat{p}-p_0}{\sqrt{\frac{p_0(1-p_0)}{n}}}=0.783$
对于右侧检验,$\alpha = 0.05$ 对应的正态分布临界值为 $Z_{0.05} = \mathbf{1.645}$,
因为 $Z = 0.783 < 1.645$,观察值没有落在拒绝域内,
在 $\alpha = 0.05$ 的水平下,不能拒绝原假设 $H_0$。

虽然样本废品率($0.025$)略高于规定的 $0.02$,但在统计学上这种差异并不显著,
可能是由于随机抽样误差引起的。因此,这批产品可以被接收
\end{solution}

\begin{remark}
小样本t分布关注方差未知且总体服从正态,大小样本卡方分布总体服从正态,大样本Z分布总体方差已知。
Z分布也是对称,左侧检验0.05显著性水平就是-1.645,右侧检验0.05显著性水平就是1.645。
\end{remark}

\begin{exercise}[非参数拟合优度检验]
某种配偶的后代按体格的属性分为三类,各类的数目是:10,53,46. 按照某种遗传模型其频率比应为 
$p^2{:}2p(1-p){:}(1-p)^2$,求问在$\alpha=0.05$水平下数据与模型是否相符?
\end{exercise}

\begin{solution}
$H_0$:观测频数符合遗传模型 $p^2 : 2p(1-p) : (1-p)^2$,
$H_1$:观测频数不符合该遗传模型。

已知观测频数 $n_1=10, n_2=53, n_3=46$,总样本量 $n=109$,构造似然函数 $L(p)$,
$L(p) = (p^2)^{n_1} \cdot [2p(1-p)]^{n_2} \cdot [(1-p)^2]^{n_3}$,
$\ln L(p) = 2n_1 \ln p + n_2 (\ln 2 + \ln p + \ln(1-p)) + 2n_3 \ln(1-p)$,
$\frac{d\ln L(p)}{dp}=\frac{2n_1+n_2}{p}-\frac{n_2+2n_3}{1-p}=0$,
$\hat{p} = \frac{73}{73 + 145} = \frac{73}{218} \approx \mathbf{0.3349}$

$P_1 = \hat{p}^2$ $\approx 0.1122$,$P_2 = 2\hat{p}(1-\hat{p})$ $\approx 2 \times 0.3349 
\times 0.6651 \approx 0.4455$,$P_3 = (1-\hat{p})^2$ $\approx 0.6651^2 \approx 0.4424$

$\chi^2 = \sum \frac{(O_i - E_i)^2}{E_i}$,$\chi^2 = \frac{(10 - 12.23)^2}{12.23} + 
\frac{(53 - 48.56)^2}{48.56} + \frac{(46 - 48.22)^2}{48.22}$,$\chi^2 \approx 0.406 
+ 0.406 + 0.102 = \mathbf{0.914}$

类别数 $k=3$。估计参数个数 $m=1$(估计了 $p$)。$df = (k-1) - m = (3-1) - 1 = \mathbf{1}$
,查表得 $\chi_{0.05}^2(1) = \mathbf{3.841}$,$\chi^2 = 0.914 < 3.841$。
结论:接受原假设 $H_0$,观测数据与遗传模型相符。
\end{solution}

\begin{remark}
标准化后的偏差 $Z_i = \frac{O_i - E_i}{\sqrt{E_i}}$ 近似服从标准正态分布 $N(0, 1)$,
相符是右侧检验
\end{remark}

\begin{exercise}[非参数拟合优度检验]
从自动精密机床产品传递带中取出 200 个零件,以 1$\mu m$以内的测量值检验零件尺寸,把测量与额定尺
寸按每隔 5$\mu m$进行分组,这种偏差落在各组内的频数$n$:如下表,试问尺寸偏差是否服从正态分布
$(\alpha=0.05)?$

\begin{center}
\begin{tabular}{c|cccccccccc}
\text{组号} & 1 & 2 & 3 & 4 & 5 & 6 & 7 & 8 & 9 & 10 \\
\hline
\text{组下限} & -20 & -15 & -10 & -5 & 0 & 5 & 10 & 15 & 20 & 25 \\
\text{组上限} & -15 & -10 & -5 & 0 & 5 & 10 & 15 & 20 & 25 & 30 \\
$n_i$ & 7 & 11 & 15 & 24 & 49 & 41 & 26 & 17 & 7 & 3 \\
\end{tabular}
\end{center}
\end{exercise}

\begin{solution}
原假设 $H_0$:零件的尺寸偏差服从正态分布 $N(\mu, \sigma^2)$,
备择假设 $H_1$:零件的尺寸偏差不服从正态分布。显著性水平:$\alpha = 0.05$

估计总体参数 ($\mu$ 和 $\sigma$)由于题目没有给出总体的均值和方差,我们需要用样本数据进行估计。
取各组的组中值($x_i$)来代表该组数据。

样本均值:$\bar{x} = \frac{\sum n_i x_i}{n} = \frac{860}{200} = 4.3$,样本标准差:
$s^2 = \frac{\sum n_i x_i^2 - n\bar{x}^2}{n-1} = \frac{22550 - 200 \times 4.3^2}{199} = 
\frac{22550 - 3698}{199} \approx 94.73$,$s = \sqrt{94.73} \approx 9.73$,
由此假设总体服从 $N(4.3, 9.73^2)$。

计算各组的理论频数 ($E_i$)
利用标准正态分布公式 $Z = \frac{X - \mu}{\sigma}$ 将各组边界标准化,查标准正态分布表 
$\Phi(Z)$ 计算概率 $p_i$,进而得到理论频数 $E_i = n \times p_i$

合并小频数并计算卡方值卡方检验要求各组理论频数不宜过小(一般要求 $E_i \ge 5$)。第1组 
$E_1=4.78 < 5$,与第2组合并。第10组 $E_{10}=3.32 < 5$,与第9组合并。

合计$\chi^2 = 6.857$,$df = 8 - 1 - 2 = 5$,$\chi_{0.05}^2(5) = 11.0705$,
$\chi^2 = 6.857 < 11.0705$,统计量未落入拒绝域,不能拒绝原假设 $H_0$,
在 $\alpha=0.05$ 的显著性水平下,认为该零件尺寸偏差服从正态分布
\end{solution}

\begin{remark}
记住构造卡方的公式:$\frac{(O_i-E_i)^2}{E_i}$,其中$E_i$为频数,概率乘以个数。
\end{remark}

\subsection{重难点学习}

重点关注t分布和样本方差构造卡方分布

1.样本及抽样,经验函数

样本:总体中抽取的部分个体称为样本。抽样:为推断总体分布及各种特征,按一定规则从总体中抽取若干个
体进行观察试验,以获得有关总体的信息,这一抽取过程称为抽样。

经验函数就是从小到大排列样本值,然后计算每个样本出现的频率。

2.统计量

如果样本$\mathbf{X}_1,\mathbf{X}_2,...,\mathbf{X}_n$的函数$g(\mathbf{X}_1,\mathbf{X}_2,...,
\mathbf{X}_n)$不含有任何的未知参数,则称函数$g(\mathcal{X}_1,\mathcal{X}_2,\ldots,
\mathcal{X}_n)$为统计量。

几个常用的统计量:样本均值$\overline{X}=\frac{1}{n}\sum_{i=1}^nX_i$,样本方差:
$S^2=\frac{1}{n-1}\sum_{i=1}^n(X_i-\overline{X})^2$,样本k阶原点矩:
$A_k=\frac{1}{n}\sum_{i=1}^nX_i^k$,样本k阶中心矩:
$B_k=\frac{1}{n}\sum_{i=1}^n(X_i-\overline{X})^k$。

3.抽样分布,$\chi^{2}$分布,t分布,F分布,分位数,抽样分布定理

$\chi^{2}$分布:$X_1,X_2,...,X_n$相互独立,都服从标准正态分布,$N(0,1)$, 则称随机变量
$\chi^2={X_1}^2+{X_2}^2+\cdots+{X_n}^2$服从自由度为n的卡方分布,记为$\chi^2(n)$。
$E(\chi^2)=n,D(\chi^2)=2n$,卡方分布具有可加性。

t分布:$X{\sim}N(0,1),Y{\sim}\chi^2(n)$,且XY相互独立,则称变量:$t=\frac{X}{\sqrt{Y/n}}$服从
的分布为自由度为 n 的 t 分布.记为$t~t(n)$。t分布关于y轴对称。当自由度n足够大时,t分布近似于
标准正态分布。

F分布:$X\sim\chi^2(n_1),Y\sim\chi^2(n_2)$,且XY相互独立,则称统计量$F=\frac{X/n_1}{Y/n_2}$
服从自由度为$n_1$及$n_2$的F分布,记作$F{\sim}F(n_1,n_2)$,$\frac{1}{F}=\frac{Y/n_2}{X/n_1}
\sim F(n_2,n_1)$

上$\alpha$分位点:$P\{X\geq x_\alpha\}=\int_{x_\alpha}^{+\infty}f(x)dx=\alpha$

抽样分布定理:样本均值满足$\overline{X}\sim N(\mu,\frac{\sigma^2}{n})$,样本均值与样本方差
相互独立。$\frac{(n-1)S^2}{\sigma^2}=\frac{\sum_{i=1}^n(X_i-\overline{X})^2}{\sigma^2}
\sim\chi^2(n-1)$,$\frac{\overline{X}-\mu}{S/\sqrt{n}}\sim t(n-1)$

二点分布:$E(X)=p\quad E(X^2)=p\quad D(X)=p(1-p)$,二项分布:$X{\sim}\mathbf{B}(n,p)$,
$E(X)=np,D(X)=np(1-p)$,泊松分布:$\mathbf{X}{\sim}\pi(\lambda)$,
$P\{X=k\}=\frac{\lambda^ke^{-\lambda}}{k!},\quad k=0,1,2,\cdots$,
$E(X)=\lambda,D(X)=\lambda$,均匀分布:$\mathbf{X}{\sim}\mathbf{U}(a,b)$,
$E(X)=\frac{a+b}{2},D(X)=\frac{(b-a)^2}{12}$,正态分布:$\mathbf{X}{\sim}N\left(\mu,\sigma^{2}\right)$,
$E(X)=\mu,D(X)=\sigma^2$,指数分布:$E(X)=\theta,D(X)=\theta^2$

4.参数估计:矩估计,最大似然估计

矩估计:理论依据辛钦大数定律及其推论。令样本均值等于期望就是$\mu$的矩估计,令平方均值等于
二阶矩就是$\sigma$的矩估计。

极大似然估计法:似然函数:$L(\theta_1,\theta_2,\cdotp\cdotp\cdotp\theta_k)=\prod_{i=1}^np
(x_i,\theta_1,\theta_2,\cdotp\cdotp\cdotp\theta_k)$,取对数然后求导即可。

5.估计量的评选标准:无偏性,有效性,Rao-Cramer定理(不要求记忆公式),一致性(大数定理)

无偏性:设 $\hat{\theta}$ 是未知参数$\theta$的估计量,若 $E(\hat{\theta})=\theta$,则称
$\hat{\boldsymbol{\theta}}$ 是$\theta$的无偏估计量。

有效性:方差最小的无偏估计量

Rao-Cramer:$D(\hat{\theta})\geq\frac{1}{nE\left[\frac{\partial}{\partial\theta}\ln p(X,\theta)
\right]^2}=D_0(\theta)$

一致性:定义 设 $\hat{\theta}=\hat{\theta}(X_1,X_2,\cdots,X_n)$ 是总体参数$\theta$的估计量。
若对于任意的$\theta\in\Theta$,当n→∞时$, \hat{\boldsymbol{\theta }}$ 依概率收敛于θ,即对于任意正数
$\varepsilon$,有 $\lim _n\to \infty P( \left | \hat{\theta } - \theta \right . ) \left | 
\geq \varepsilon \right ) = 0$,则称$\hat{\boldsymbol{\theta}}$是总体参数$\theta$的一致(或相合)估计量

一般,矩估计法得到的估计量为一致估计量

6.单个正态总体的均值,方差区间估计:双侧置信区间,单侧置信区间

双侧置信区间与置信度:设总体 $X$ 的分布函数 $F(x; \theta)$ 含有一个未知参数 $\theta$,对给定的值 
$\alpha (0 < \alpha < 1)$,如果有两个统计量 $\hat{\theta}_1 = \hat{\theta}_1(X_1, \cdots, X_n)$,
$\hat{\theta}_2 = \hat{\theta}_2(X_1, \cdots, X_n)$,使得:$P\left\{ \hat{\theta}_1(X_1, \cdots, X_n) 
\leq \theta \leq \hat{\theta}_2(X_1, \cdots, X_n) \right\} \geq 1 - \alpha \quad \forall \theta \in 
\Theta \quad (7-1)$,则称随机区间 $\left( \hat{\theta}_1, \hat{\theta}_2 \right)$ 是 $\theta$ 的双侧 
$1 - \alpha$ 置信区间;称 $1 - \alpha$ 为置信度;$\hat{\theta}_1$ 和 $\hat{\theta}_2$ 分别称为双侧置信
下限和双侧置信上限。

在以上定义中,若将(7-1)式改为:$P\left\{\hat{\theta}_{1}\left(X_{1}, \cdots, X_{n}\right) \leq \theta
\right\} \geq 1-\alpha, \quad \forall \theta \in \Theta \quad (7-2)$,则称$\hat{\theta}_{1}\left(X_{1}
, \cdots, X_{n}\right)$为$\theta$的单侧置信下限。随机区间$\left(\hat{\theta}_{1},+\infty\right)$是
$\theta$的置信度为$1-\alpha$的单侧置信区间。又若将(7-2)式改为:$P\left\{\theta \leq \hat{\theta}_{2}
\left(X_{1}, \cdots, X_{n}\right)\right\} \geq 1-\alpha, \quad \forall \theta \in \Theta \quad (7-3)$,
则称$\hat{\theta}_{2}\left(X_{1}, \cdots, X_{n}\right)$为$\theta$的单侧置信上限。
随机区间$\left(-\infty, \hat{\theta}_{2}\right)$是$\theta$的置信度为$1-\alpha$的单侧置信区间。

单个正态总体方差区间估计:

$\mu$的置信区间

$\begin{aligned} \begin{pmatrix} 1 \end{pmatrix} & \sigma^2\text{已知时} 
\\ & \overline{X}\text{是}\mu\text{的无偏估计,由 }\frac{\overline{X}-\mu}{\sigma/\sqrt{n}}\sim 
N\left(0,1\right) \\ & \text{有}P\left\{\left|\frac{\overline{X}-\mu}{\sigma/\sqrt{n}}\right|
<Z_{\alpha/2}\right\}=1-\alpha \\ & \text{即}P\left\{\overline{X}-\frac{\sigma}{\sqrt{n}}Z_{\alpha/2}
<\mu<\overline{X}+\frac{\sigma}{\sqrt{n}}Z_{\alpha/2}\right\}=1-\alpha \\ & & \text{置信区间为:}\boxed
{\left(\bar{X}-\frac{\sigma}{\sqrt{n}}Z_{\alpha/2},\bar{X}+\frac{\sigma}{\sqrt{n}}Z_{\alpha/2}\right)} 
\end{aligned}$

$\begin{aligned} \left(2\right)\sigma^2\text{未知时} \\ & \text{由 }\frac{\bar{X}-\mu}{S/\sqrt{n}}
\sim t\left(n-1\right) \\ & \text{有}P\left\{-t_{\alpha/2}\left(n-1\right)<\frac{\overline{X}-\mu}
{S/\sqrt{n}}<t_{\alpha/2}\left(n-1\right)\right\}=1-\alpha \\ & \text{即}P\left\{\overline{X}-\frac{S}
{\sqrt{n}}t_{\alpha/2}\left(n-1\right)<\mu<\overline{X}+\frac{S}{\sqrt{n}}t_{\alpha/2}\left(n-1\right)
\right\}=1-\alpha \\ & \text{置信区间为:}\boxed{\left(\overline{X}-\frac{S}{\sqrt{n}}t_{\alpha/2}
\left(n-1\right),\overline{X}+\frac{S}{\sqrt{n}}t_{\alpha/2}\left(n-1\right)\right)} \end{aligned}$

方差$\sigma^2$的置信区间

$\begin{aligned} & \text{设μ未知} \\ & & & \text{由 }\frac{\left(n-1\right)S^2}{\sigma^2}\sim\chi^2
\left(n-1\right) \\ & & & \text{有}P\left\{\chi_{1-\alpha/2}^2\left(n-1\right)<\frac{\left(n-1\right)
S^2}{\sigma^2}<\chi_{\alpha/2}^2\left(n-1\right)\right\}=1-\alpha \\ & & & \text{即}P\left\{\frac
{\left(n-1\right)S^2}{\chi_{\alpha/2}^2\left(n-1\right)}<\sigma^2<\frac{\left(n-1\right)S^2}{\chi_
{1-\alpha/2}^2\left(n-1\right)}\right\}=1-\alpha \\ & \text{置信区间为:}\boxed{\left(\frac
{\left(n-1\right)S^2}{\chi_{\alpha/2}^2\left(n-1\right)},\frac{\left(n-1\right)S^2}{\chi_{1-\alpha/2}
^2\left(n-1\right)}\right)} \end{aligned}$

参数检验最后一页有全部分布公式

7.假设检验:单一正态总体的均值,方差的假设检验(单侧,双侧)

均值($\sigma^2$已知)单侧与双侧:$\begin{array} {ccccc}\text{类型} & \text{原假设} & \text{备择假设} & \text{检验统计量} 
& \text{拒绝域} \\ & H_0 & H_1 & \text{检验统计量} & \text{拒绝域} \\ \text{双边} & \mu=\mu_0 & 
\mu\neq\mu_0 & & |U|\geq z_{\frac{\alpha}{2}} \\ & & & U=\frac{\overline{X}-\mu_0}{\sigma_0/
\sqrt{n}} & \\ \text{单边} & \mu\leq\mu_0 & \mu>\mu_0 & & U\geq z_\alpha \\ \text{检验} & & & & \\ & 
\mu\geq\mu_0 & \mu<\mu_0 & & U\leq-z_\alpha \end{array}$

均值($\sigma^2$未知)单侧与双侧:$\begin{array} {ccccc}\text{类型} & \text{原假设} & \text{备择假设} & 
& \\ & H_0 & H_1 & \text{检验统计量} & \text{拒绝域} \\ \text{双边} & \mu=\mu_0 & \mu\neq\mu_0 & 
& |T|\geq t_{\frac{a}{2}}(n-1) \\ \text{检验} & & & T=\frac{\overline{X}-\mu_0}{S/\sqrt{n}} & 
T\geq t_a(n-1) \\ \text{单边} & \mu\leq\mu_0 & \mu>\mu_0 & & \\ \text{检验} & & & & \\ & \mu\geq\mu_0 
& \mu<\mu_0 & & T\leq-t_a(n-1) \end{array}$

方差($\mu$已知):$\begin{array} {ccccc}\text{类型} & \text{原假设} & \text{备择假设} & 
\text{检验统计量} & \text{拒绝域} \\ & H_0 & H_1 & & \\ \text{双边} & \sigma^2=\sigma_0^2 & \sigma^2
\neq\sigma_0^2 & \chi^2= & \chi^2\leq\chi_{1-\frac{\alpha}{2}}^2(n) \\ \text{检验} & & & \chi^2=
\chi_{\frac{\alpha}{2}}^2(n) \\ \text{单边} & \sigma^2\leq\sigma_0^2 & \sigma^2>\sigma_0^2 & \sum_
{i=1}^n\left(\frac{X_i-\mu_0}{\sigma_0}\right)^2 & \chi^2\geq\chi_\alpha^2(n) \\ \text{检验} & & & & 
\chi^2\leq\chi_{1-\alpha}^2(n) \\ & \sigma^2\geq\sigma_0^2 & \sigma^2<\sigma_0^2 & & \chi^2\leq\chi_
{1-\alpha}^2(n) \end{array}$

方差($\mu$未知):$\begin{array} {ccccc}\text{类型} & \text{原假设} & \text{备择假设} & 
\text{检验统计量} & \text{拒绝域} \\ & H_0 & H_1 & & \\ & & & & \chi^2\leq\chi_{1-\frac{a}{2}}^2(n-1) 
\\ \text{双边} & \sigma^2=\sigma_0^2 & \sigma^2\neq\sigma_0^2 & & \\ \text{检验} & & & \chi^2=\frac
{(n-1)S^2}{\sigma_0^2} & \chi^2\geq\chi_2^2(n-1) \\ \text{单边} & \sigma^2\leq\sigma_0^2 & \sigma^2>
\sigma_0^2 & & \\ \text{检验} & & & & \\ & \sigma^2\geq\sigma_0^2 & \sigma^2<\sigma_0^2 & & \chi^2
\leq\chi_{1-a}^2(n-1) \end{array}$

\section{总结}

\subsection{矩阵论}
坐标,向量,基底公式:$Ak=x$

基变换公式:$B_2=B_1C$

子空间问题:$w_1+w_2$为线性组合找基或者维数直接找组成他们的原向量的最大无关组,$w_1\bigcap w_2$找公共解
同时利用公式$dim(W_1+W_2)=dimW_1+dimW_2-dim(W_1\bigcap W_2)$,一般能得到$w_1\bigcap w_2=1$

证明直和问题:先说明交集为0,推出$V_1+V_2=V_1\bigoplus V_2$,然后说明$dim(V_1+V_2)$的维数为$n$。

线性变换:需要验证加法和乘法。线性变换T在基下的矩阵:$T(B)=BA$

线性变换T的特征值和特征向量:先求A特征值(即T特征值),再求A特征向量(T特征向量的坐标),最后求T特征向量
代入公式$Ak=x$。

最小多项式:约旦阵最小多项式是其本身,分块为各自的公因式,普通矩阵直接按照$A-I,(A-I)^2,...$找到首个等于0为止。
多个特征值时,也可以挨个试,但是要记住把所有特征值的多项式乘起来。

约旦标准型求法:先求A的特征值,再求A的特征向量$(A-\lambda I)\xi_n=\xi_{n-1}$,然后将所有特征向量组合
就是P阵,约旦阵按照P对应的特征值排列。

微分方程组初始条件解:求$e^{At}$,注意辨别先利用约旦型求的状态转移矩阵是$e^{Jt}$
再代入公式:$\mathbf{x}(t)=e^{At}\mathbf{x}(0)+\int_0^te^{A(t-\tau)}
\mathbf{f}(\tau)d\tau$

满秩分解:行最简后的C,取主元列原矩阵得B

QR分解:施密特正交化(投影),$A=QR$,$R=Q^TA$

SVD分解:$A^HA$求奇异值以及正交化后的特征向量得V,$AA^H$求正交化后的特征向量得U。$A=U\Sigma V^H$

酉矩阵是实数域上的正交矩阵,正规矩阵$A^H A = A A^H$

\subsection{数值分析}

拉格朗日插值函数:$L_n(x) = \sum_{i=0}^{n} y_i \cdot l_i(x)$,$l_i(x) = \prod_{j=0, j \neq i}^{n} 
\frac{x - x_j}{x_i - x_j}$,差商和导数间存在联系:$f[x_0, x_1, \cdots, x_n] = \frac{f^{(n)}(\xi)}{n!}$

牛顿插值:构造差商表即可,可以按x大小排列。

插值误差:套余项公式即可。$R_n(x) = f(x) - P_n(x) = \frac{f^{(n+1)}(\xi)}{(n+1)!} \prod_{i=0}^{n} 
(x - x_i)$,分段线性插值误差公式:$|f(x) - I_h(x)| \le \frac{1}{8} \cdot \max_{a \le x \le b} 
|f''(x)| \cdot h^2$

Hermite插值多项式:扩展牛顿插值,引入导数项。

最佳二次平方逼近:选取基底构造法方程组。

Simpson公式:$ \int_{a}^{b} f(x) dx \approx \frac{b-a}{6} [f(a) + 4f(\frac{a+b}{2}) + f(b)]$。误差:
$E_s = -\frac{(b-a)^5}{2880} f^{(4)}(\xi)$

梯形公式:$ \int_{a}^{b} f(x) dx \approx \frac{b-a}{2} [f(a) + f(b)]$。误差:$E_t = -\frac{(b-a)^3}{12} f''(\xi)$

复化Simpson公式:$S_n = \frac{h}{3} \left[ f(a) + 4\sum_{k=\text{odd}}^{n-1} f(x_k) + 2\sum_
{k=\text{even}}^{n-2} f(x_k) + f(b) \right]$,误差:$R_S = -\frac{b-a}{180} h^4 f^{(4)}(\xi) = O(h^4)$

复化梯形公式:$T_n = \frac{h}{2} \left[ f(a) + 2\sum_{k=1}^{n-1} f(x_k) + f(b) \right]$。误差:
$R_T = -\frac{b-a}{12} h^2 f''(\xi) = O(h^2)$

Gauss型求积公式:精度为2n-1,n为插值点个数。

Gauss-Legendre和Gauss-Chebyshev公式:需要先将积分转到[-1,1]区间。勒得让型只需要套公式节点和权重即可;切比
雪夫需要公式$\int_{-1}^{1} \frac{f(t)}{\sqrt{1-t^2}} dt \approx \frac{\pi}{n} \sum_{k=1}^{n} f(t_k)$
,$t_k = \cos\left(\frac{2k-1}{6}\pi\right)$。

改进欧拉法:根据当前值,算当前斜率,然后算出预估步,利用预估步算出校正斜率,最后取均值。

精度问题:泰勒展开,放大因子绝对值小于1算法绝对稳定。

二阶龙格库塔:$\begin{cases}K_1 = f(x_n, y_n) \\K_2 = f(x_n + h, y_n + h K_1) \\
  y_{n+1} = y_n + \frac{h}{2}(K_1 + K_2)\end{cases}$

四阶龙格库塔:$\begin{cases}
K_1 = f(x_n, y_n) \\K_2 = f(x_n + \frac{h}{2}, y_n + \frac{h}{2}K_1) \\
K_3 = f(x_n + \frac{h}{2}, y_n + \frac{h}{2}K_2) \\K_4 = f(x_n + h, y_n + h K_3) \\
y_{n+1} = y_n + \frac{h}{6}(K_1 + 2K_2 + 2K_3 + K_4)\end{cases}$

范数和条件数:无穷范数行范,1范列范,2范最大奇异值,F范平方和开根号;条件数为$\kappa(A) = ||A|| 
\cdot ||A^{-1}||$。A或b有误差时,存在相对误差,公式看前面备注。

Jacobi迭代:$A=L+D+U$

Gauss-Seidel迭代:$A=L+D+U$,U算作前一步

迭代法的收敛性:迭代矩阵的特征值绝对值最大的小于1(谱半径)。

非线性的迭代:牛顿法或者不动点法。$J(\mathbf{x}^{(k)}) \Delta \mathbf{x}^{(k)} = -F(\mathbf{x}^{(k)})$

不动点迭代直接对$x_k$的部分构造公式求解,求导与初值根位置判断。

Newton法则写出求根形式构造出$f(x)$然后套公式即可。

\subsection{数理统计}

卡方分布,t分布,F分布。

假设检验的几个构造公式,看课件。

\begin{center}
\Huge 考试顺利!!!
\end{center}
\end{document}